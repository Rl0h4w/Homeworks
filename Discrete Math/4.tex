\documentclass[a4paper,12pt]{article}

% Кодировка и язык
\usepackage[utf8]{inputenc}
\usepackage[russian]{babel}

% Математические пакеты
\usepackage{amsmath,amsfonts,amssymb}

% Графика
\usepackage{graphicx}
\usepackage{tikz}
\usetikzlibrary{shapes.geometric, calc}
\usepackage{pgfplots}
\pgfplotsset{compat=1.18} % Добавлено для устранения предупреждения PGFPlots

% Геометрия страницы
\usepackage{geometry}
\geometry{top=2cm, bottom=2cm, left=2.5cm, right=2.5cm}

% Гиперссылки
\usepackage{hyperref}

% Плавающие объекты
\usepackage{float}

% Дополнительные пакеты
\usepackage{venndiagram}

% Настройки заголовка
\title{Домашнее задание}
\author{Студент: \textbf{Ростислав Лохов}}
\date{\today}

\begin{document}

% Титульный лист
\begin{titlepage}
    \centering
    \vspace*{1cm}

    \Huge
    \textbf{Домашнее задание}

    \vspace{0.5cm}
    \LARGE
    По курсу: \textbf{Дискретная математика}

    \vspace{1.5cm}

    \textbf{Студент: Ростислав Лохов}

    \vfill

    \Large
    АНО ВО Центральный университет\\
    \vspace{0.3cm}
    \today

\end{titlepage}

% Содержание
\tableofcontents
\newpage

% Основной текст
\section{Дирихле, оценка+пример}


\subsection{Задача 1}
\begin{enumerate}
    \item Т.к прямая - $\mathbb{R}$, множество расстояний между всеми элементами - $\mathbb{R}$ нас интересует только подмножество целых чисел.
    \item Т.к мощность множества целых чисел больше множества цветов, то по принципу дирихле найдется две различные точки, что расстояние между ними целое число.
\end{enumerate}

\subsection{Задача 2}
\begin{enumerate}
    \item Рассмотрим множество остатков деления на 100: его мощность - 100
    \item Всего чисел 100, а значит сумма 100 чисел делится на 100 по принципу дирихле
\end{enumerate}

\subsection{Задача 3}
\begin{enumerate}
    \item Пример: {1, 1, 1, 2, 2, 2, 2} - 1 неделя, упорядоченное множество по дням недели, каждый день не менее 1, сумма - 11 -- Подходит
    \item Вторая неделя - {2, 2, 2, 2, 1, 1, 1} - 2 неделя, упорядоченное множество по дням недели, каждый день не менее 1 задачи, сумма 11 -- Подходит
    \item тогда со вторника первой недели включительно до сб первой недели включительно будет ровно 20 задач решено.
\end{enumerate}

\subsection{Задача 4}
\begin{enumerate}
    \item $AB$ - два цвета, значит должно для двух фишек цвета А и B найтись две соседние фишки этих цветов - нашлись $AB$. Мощность ряда - 2. Значит всего 2 фишки. т.к меньше мы уже не сможем выбрать меньше, чем 2 цвета.
\end{enumerate}

\subsection{Задача 5}
\begin{enumerate}
    \item рассмотрим случай, где максимальное кол-во одинаковых элементов, тогда будет последовательность вида $0123456789+...$ в таком случае минимальная допустимая мощность множества - 10. Иначе не все цвета будут представлены
    \item рассмотрим, когда мы добавили 1 элемент, допустим 0, тогда $01234567890$ тогда возможен такой случай, где мы выбрали два 0 и всё остальное. Значит гарантированно мы можем вытащить 9 максимальных элементов. Продолжая таким образом и добавляя всегда один и тот же элемент, мы получаем, что максимальное количество - 15. 
    \item хорошо, рассмотрим в таком случае $123456789000000$ тогда возможен случай, что мы выбрали $6789000000$ - 5 уникальных шаров, т.е гарантированно мы получим минимум 5 уникальных шаров при условии, что вместо 4х 0 будет что-то другое из набора 10 цветов.
    \item добавим следующий элемент, пусть будет также 0, $123456789000000$ и выбором из 10 возможен вариант $789000000$ т.е 4 уникальных шара. Что будет минимумом, т.к мы максимизировали кол-во повторяющихся элементов одного цвета.
    \item 15 маркеров
\end{enumerate}

\subsection{Задача 6}
\begin{enumerate}
    \item Предположим обратное, не существует двух людей с совпадаю
    \item Рассмотрим множество людей с одинаковыми фамилиями ${2, 3, 4,...11}$ не факт, что конкретно такое множество, главное знать, что хотя бы какой то опорный элемент из этого множества существует.
    \item Допустим взяли произвольное число, допустим 4 из этого множества. Оно означает сколько всего людей с такой фамилией.
    \item Далее мы берем произвольного человека из этого множества. Тогда у этого человека очевидно, что существует имя. 
    \item По условию задачи, среди 66 написаных чисел встречается каждое из 1-10. Т.е для нашего человека могут быть 1-10 однофамильцев.
    \item Тогда в группе есть 2 человека с совпадающими именем и фамилией.
\end{enumerate}

\subsection{Задача 8}
\begin{enumerate}
    \item на весь ряд не менее 13 сосен и 25 берез. Тогда дубов не более 112
    \item Приведем пример: Березы: 6, 12, 18, ... 144, 150; Сосны: 11, 22, ..., 55, 65(сдвиг для того, чтобы не совпадали с соснами), 76, 119, 130, 141.
    \item Тогда 25 берез 13 сосен и 112 дубов.
\end{enumerate}

\end{document}