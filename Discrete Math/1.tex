\documentclass[a4paper,12pt]{article}

% Кодировка и язык
\usepackage[utf8]{inputenc}
\usepackage[russian]{babel}

% Математические пакеты
\usepackage{amsmath,amsfonts,amssymb}

% Графика
\usepackage{graphicx}
\usepackage{tikz}
\usetikzlibrary{shapes.geometric, calc}
\usepackage{pgfplots}
\pgfplotsset{compat=1.18} % Добавлено для устранения предупреждения PGFPlots

% Геометрия страницы
\usepackage{geometry}
\geometry{top=2cm, bottom=2cm, left=2.5cm, right=2.5cm}

% Гиперссылки
\usepackage{hyperref}

% Плавающие объекты
\usepackage{float}

% Дополнительные пакеты
\usepackage{venndiagram}

% Настройки заголовка
\title{Домашнее задание}
\author{Студент: \textbf{Ростислав Лохов}}
\date{\today}

\begin{document}

% Титульный лист
\begin{titlepage}
    \centering
    \vspace*{1cm}

    \Huge
    \textbf{Домашнее задание}

    \vspace{0.5cm}
    \LARGE
    По курсу: \textbf{Дискретная математика}

    \vspace{1.5cm}

    \textbf{Студент: Ростислав Лохов}

    \vfill

    \Large
    АНО ВО Центральный университет\\
    \vspace{0.3cm}
    \today

\end{titlepage}

% Содержание
\tableofcontents
\newpage

% Основной текст
\section{Булева алгебра}

ДИСКЛЕЙМЕР: Мне очень нужно попасть на красно-черный уровень, для этого нужно 8  баллов минимум набрать в течение 3х недель, не придирайтесь строго пожалуйста:) Я попал сюда потому что пропустил тест:(

\subsection{Задача 1}
\begin{enumerate}
    \item Если он синий, то он круглый ( из либо синий либо желтый)
    \item Если он желтый, то квадратный, если он квадратный, то он красный - противоречеие
    \item Значит синий круглый(пооооооолукруг голубоооооооооой)
\end{enumerate}

\subsection{Задача 2}
\begin{enumerate}
    \item Рассмотрим первого, если он лжец, тогда вокруг него два рыцаря, если он рыцарь - два лжеца
    \item Если второй лжец, тогда вокруг него него либо нет лжецов, либо два лжеца, если рыцарь - действительно один 
    \item тогда если третий рыцарь, то он может сказать, что вокруг него один лжец - первый.
\end{enumerate}

\subsection{Задача 3}
\begin{enumerate}
    \item $(C\lor R) \to \overline{M} = (\overline{C}\land \overline{R}) \lor \overline{M} = (\overline{C}\lor \overline{M}) \land (\overline{R} \lor \overline{M}) = (C \to \overline{M}) \lor (R \to \overline{M})$ т.е совпадает с пунктом 4
    \item $M \to (\overline{C} \land \overline{R}) = \overline{M}\lor\overline{C\lor R} = M \land C \land R$ = совпадает с 1ым т.к если одно условие не выполняется, то мероприятия нет
    \item $\overline{M} \to (C \lor R) = M \lor (\overline{C}\land\overline{R})$
    \item $R \to \overline{M} \land C \to \overline{M}$
    \item $\overline{C} \land \overline{R} \to M = (C \lor R) \lor M$ - совпадает с 3м
    \item т.е АБГ и ВД
\end{enumerate}

\subsection{Задача 4}
Для начала упростим:
\[
(\overline{A} \lor C) \land (\overline{A} \lor \overline{C}) \land (A\lor (C \land \overline{B} \land D))
\]

\[
(\overline{A} \lor C) \land (\overline{A} \lor \overline{C}) \land ((A\lor C) \land (A \lor \overline{B}) \land (A \lor D))
\]

\[
\overline{A} \land C \land \overline{B} \land D
\]


\[
A = 0; B = 0; C = 1; D = 1
\]

\subsection{Задача 5}
Просто составим функцию: 

\[ f(x) = 
\begin{cases}
    14 > x, \text{если} x > 16\\
    14 < x, \text{если} x < 16
\end{cases}
\]
Получим, что функция истинна только если x = 15

\subsection{Задача 6}
\begin{enumerate}
    \item $(P\land\overline{Q}) \lor (P\land \overline{R}) = P \land \overline{Q} \land \overline{R}$
    \item $(P\lor \overline{Q}) \lor (P\land \overline{R}) = (P\lor P \lor \overline{Q})\land (\overline{R} \lor P \lor \overline{Q})=(P\lor \overline{Q})\land (\overline{R}\lor P \lor \overline{Q}) = (P \land (\overline{R}\lor P \lor \overline{Q}))\lor(\overline{Q} \land (\overline{R}\lor P \lor \overline{Q})) = P\lor \overline{Q}$
    \item $(P\land R) \lor (\overline{R}\land P) \lor (\overline{R} \land Q)= P \lor (\overline{R}\land Q)$
\end{enumerate}

\subsection{Задача 7}
$A\lor(B\land C)$ Не знаю что нужно пояснить, просто посмотрел сначала на второй и третий столбец как они мебя ведут, потом на первый и второй, затем обьединил и проверил

\subsection{Задача 8}
Тут через круги эйлера, не буду утруждать себя кодом в латехе для создания. Представляю в голове области и жестко пишу формулы в латехе для решения задачи:
\begin{enumerate}
    \item Заметим, что $A\oplus B \equiv (A\oplus B)\oplus (A\land B)$ тогда $(A\land B)\land C = ((A \oplus B) \oplus (A \land B)) \oplus C \oplus (((A \oplus B) \oplus (A \land B)) \land C)$ 
    \item Заметим, что $A \lor B = (\overline{A} \to B)\Rightarrow (A\land B)\lor C = \overline{A\land B} \to C = ((A \to \overline{B}) \to C)$
\end{enumerate}

\end{document}