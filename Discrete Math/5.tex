\documentclass[a4paper,12pt]{article}

% Кодировка и язык
\usepackage[utf8]{inputenc}
\usepackage[russian]{babel}

% Математические пакеты
\usepackage{amsmath,amsfonts,amssymb}

% Графика
\usepackage{graphicx}
\usepackage{tikz}
\usetikzlibrary{shapes.geometric, calc}
\usepackage{pgfplots}
\pgfplotsset{compat=1.18} % Добавлено для устранения предупреждения PGFPlots

% Геометрия страницы
\usepackage{geometry}
\geometry{top=2cm, bottom=2cm, left=2.5cm, right=2.5cm}

% Гиперссылки
\usepackage{hyperref}

% Плавающие объекты
\usepackage{float}

% Дополнительные пакеты
\usepackage{venndiagram}

% Настройки заголовка
\title{Домашнее задание}
\author{Студент: \textbf{Ростислав Лохов}}
\date{\today}

\begin{document}

% Титульный лист
\begin{titlepage}
    \centering
    \vspace*{1cm}

    \Huge
    \textbf{Домашнее задание}

    \vspace{0.5cm}
    \LARGE
    По курсу: \textbf{Дискретная математика}

    \vspace{1.5cm}

    \textbf{Студент: Ростислав Лохов}

    \vfill

    \Large
    АНО ВО Центральный университет\\
    \vspace{0.3cm}
    \today

\end{titlepage}

% Содержание
\tableofcontents
\newpage

% Основной текст
\section{Cвойства сочетаний. Сочетания с повторениями. Числа Каталана}


\subsection{Задача 1}
\begin{enumerate}
    \item Порядок неважен, т.е мы однозначно можем восстановить необходимое число из множества цифр длиной 5(любое множество цифр может быть неубывающим/невозрастающим).
    \item Тогда задача сводится к тому, чтобы найти Количество сочетаний с повторениями из 10 цифр по 5, для пятизначного числа.
    \item Каждая цифра может быть 5 раз, за исключением 0, он может быть 4 раза.
    \item Всего сочетаний повторениями $\binom{14}{5}$
    \item Т.к одно множество будет $\{0, 0, 0, 0, 0\}$ просто вычтем 1
    \item Тогда ответ будет $\frac{14!}{5!9!}-1=\frac{10*11*12*13*14}{1*2*3*4*5}-1=2001$
\end{enumerate}


\subsection{Задача 2}
\begin{enumerate}
    \item Рассмотрим первую связку, для нее возможно $\binom{60}{15}$ вариантов сочетания грибов.
    \item Рассмотрим вторую связку, для нее возможно $\binom{45}{15}$ вариантов сочетания грибов.
    \item Рассмотрим вторую связку, для нее возможно $\binom{30}{15}$ вариантов сочетания грибов.
    \item Рассмотрим вторую связку, для нее возможно $\binom{15}{15}=1$ вариантов сочетания грибов.
    \item Тогда нам необходимы такие варианты где нам необходимо выбрать все связки.
    \item Однако в нашем случае связки считаются упорядоченными т.к мы неявно задаём порядок рассматривая сначала первую связку, затем вторую и т.п
    \item Таким образом $n = \frac{\binom{60}{15} \cdot \binom{45}{15} \cdot \binom{30}{15}}{4!}$
\end{enumerate}


\subsection{Задача 3}
\begin{enumerate}
    \item Между каждой парой выбранных книг должно быть не менее 3х книг. Задача о барьерах.
    \item Для 10 выбранных книг у нас 9 промежутков, каждый занимает минимум 3 места
    \item $3\cdot 9$ книг занято, можем оперировать только $50-3\cdot 9$. Всего 10 интересующих книг. Тогда $\binom{23}{10}$ при этом условие минимум 3 места уже учтено.
\end{enumerate}

\subsection{Задача 4}
\[
\sum_{k=0}^{m} \binom{k}{n}\cdot\binom{m-k}{n} 
\]

\[
\binom{k}{n}\cdot \binom{m-k}{n-k} = \frac{n!}{k!(n-k)!}\cdot \frac{(n-k)!}{(m-k)!(n-m)!}=\frac{n!}{k!(m-k)!(n-m)!} = \frac{n!}{m!(n-m)!}\cdot \frac{m!}{k!(m-k)!}
\]

\[
=\binom{m}{n}\cdot \binom{k}{m}
\]

\[
\sum_{k=0}^{m} \binom{k}{n}\cdot\binom{m-k}{n-k} = \sum_{k=0}^{m} \binom{m}{n}\cdot \binom{k}{m} = \binom{m}{n} \sum_{k=0}^{m} \binom{k}{m} = \binom{m}{n}\cdot 2^m
\]


\subsection{Задача 5}

\[
(\sum_{k=0}^{n} \binom{k}{n})^2 = ((1+1)^n)^2=(1+1)^2n= \sum_{k=0}^{2n} \binom{k}{2n}
\]

\subsection{Задача 6}
\begin{enumerate}
    \item $C(n, k) = \frac{n!}{k!(n-k)!}$
    \item Функция строго возрастает при $n\ge k$
    \item $\forall k C(n,k)=m$ т.е имеет не более одного решения n для произвольного m
    \item Рассмотрим k>m. Минимальное значение C достигается при n=k и равно 1. Однако если мы возьмем n=k+1 то C=k+1 Если k+1>m то для всех $n\ge k+1 C(n, k) \ge k+1 > m$
    \item Таким оразом при $k\ge m $ нет решений
    \item Таким образом следует, что k может принимать значения только от 1 до m-1. Для каждого такого k существует не более одного m удовлетворяющее уравнению $C(n, k)=m$
    \item Следовательно общее кол-во пар $(n, k)$ для которых $C(n, k)=m$ не превышает m-1 что конечно. 
\end{enumerate}

\subsection{Задача 7}
\begin{enumerate}
    \item Воспользуемся теоремой Люка: бином нечетен тогда и только тогда, когда двоичные цифры в двоичном разложении n являются подмножеством битов m
    \item Т.е каждый бит k не превосходит соответствующий бит n в двоичной системе.
    \item если $n=\sum_{i=0}^{m}a_i2^i$ $k=\sum_{i=0}^{m}b_i2^i$ то C(n, k) нечетен тоже самое, что и $\forall i b_i \le a_i$
    \item Если $a_i=1 \Rightarrow b_i = 0 \lor b_i=1$
    \item Если $a_i=0 \Rightarrow b_i=0$
    \item Тогда для каждого $a_i=1$ есть 2 варианта выбора $b_i$, а для $a_i=0$ только один.
    \item Тогда общее количество нечентных $C(n, k)=\prod_{i=0}^{m}(1+a_i)=2^{\text{Кол-во единиц в n}}$
    \item Тогда в нашем случае $2025_{10} = 11111101001_2$
    \item Общее количество нечетных: $2^8=256$
    \item Общее количество четных - $2025+1-256=1770$ +1 - смещение по индексу.
    \item Задача гроб
\end{enumerate}

\subsection{Задача 8}
\begin{enumerate}
    \item n людей, 50 рублей стоимость кино, если дают купюру в 100 рублей - 50 рублей сдачи т.е минус 50 рублей.
    \item Далее у нас есть купюра в 100 рублей, если мы разменяли, т.е это тупо мертвый груз с которым мы ничего не можем делать
    \item Тогда у нас есть возможность добавить 50р в кассу, убавить кассу на 50р при этом добавить 100 р. при обоих случаях у нас тратится билет. Всего n билетов
    \item Можем просто убрать 100р, роли не играет, 50 р -50 рублей сделать +1 и -1 соответственно
    \item k людей имеют 50р. 
    \item Представим, что театр обладает вместимостью минимум n, тогда нам необходимо продать билеты всем людям, и чтобы касса могла разменивать. Также должно выполняться $k\ge n-k$ иначе ответ 0
    \item Тогда нам необходимо попасть в точку (n, 2k-n), n - количество проданых билетов, 2k-n - конечный баланс, нам необходимо, чтобы он был $\ge 0$
    \item Хорошо, тогда всего существует $\binom{n}{k}$ способов расставить людей.
    \item Построим биекцию между плохими траекториями, ведущими из (0, 0) в (n, 2k-n)
    \item Отразим часть траектории после этой точки относительно прямой y=-1
    \item В результате получим траекторию которая заканчивается в точке (n, 2k-n)
    \item При этом произвольная траектория из (0, 0) точно пересекает прямую $y=-1$
    \item Аналогично, найдем последнюю точку пересечения или касания с этой прямой и отразим эту часть вертикально. Получим плохую траекторию из (0, 0) в (n, 2k-n)
    \item Поскольку это биекция, количество плохих траекторий из (0, 0) в (n, 0) равно количеству произвольных траекторий из (0, 0) в (n, 2k-n-2).
    \item Чтобы добраться из точки (0, 0) в (2n, -2) необходимо выбрать k+1 шаг из n на котором траектория пойдет вниз
    \item Т.е k+1 людей с 50 рублями n-k-1 людей с 100 рублями
    \item Таким образом кол-во плохих траектори $\binom{n}{k+1}$ т.к мы выбираем k+1 позиций для людей с 50 рублями из n
    \item Тогда допустимых траекторий - разница между общим кол-вом и недопустимым 
    \item $\binom{n}{k}-\binom{n}{k+1}$
\end{enumerate}



\end{document}