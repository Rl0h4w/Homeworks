\documentclass[a4paper,12pt]{article}

% Кодировка и язык
\usepackage[utf8]{inputenc}
\usepackage[russian]{babel}

% Математические пакеты
\usepackage{amsmath,amsfonts,amssymb}

% Графика
\usepackage{graphicx}
\usepackage{tikz}
\usetikzlibrary{shapes.geometric, calc}
\usepackage{pgfplots}
\pgfplotsset{compat=1.18} % Добавлено для устранения предупреждения PGFPlots

% Геометрия страницы
\usepackage{geometry}
\geometry{top=2cm, bottom=2cm, left=2.5cm, right=2.5cm}

% Гиперссылки
\usepackage{hyperref}

% Плавающие объекты
\usepackage{float}

% Дополнительные пакеты
\usepackage{venndiagram}

% Настройки заголовка
\title{Домашнее задание}
\author{Студент: \textbf{Ростислав Лохов}}
\date{\today}

\begin{document}

% Титульный лист
\begin{titlepage}
    \centering
    \vspace*{1cm}

    \Huge
    \textbf{Домашнее задание}

    \vspace{0.5cm}
    \LARGE
    По курсу: \textbf{Дискретная математика}

    \vspace{1.5cm}

    \textbf{Студент: Ростислав Лохов}

    \vfill

    \Large
    АНО ВО Центральный университет\\
    \vspace{0.3cm}
    \today

\end{titlepage}

% Содержание
\tableofcontents
\newpage

% Основной текст
\section{Cвойства сочетаний. Сочетания с повторениями. Числа Каталана}


\subsection{Задача 1}
\begin{enumerate}
    \item Порядок неважен, т.е мы однозначно можем восстановить необходимое число из множества цифр длиной 5(любое множество цифр может быть неубывающим/невозрастающим).
    \item Тогда задача сводится к тому, чтобы найти Количество сочетаний с повторениями из 10 цифр по 5, для пятизначного числа.
    \item Каждая цифра может быть 5 раз, за исключением 0, он может быть 4 раза.
    \item Всего сочетаний повторениями $\binom{14}{5}$
    \item Т.к одно множество будет $\{0, 0, 0, 0, 0\}$ просто вычтем 1
    \item Тогда ответ будет $\frac{14!}{5!9!}-1=\frac{10*11*12*13*14}{1*2*3*4*5}-1=2001$
\end{enumerate}


\subsection{Задача 2}
\begin{enumerate}
    \item Рассмотрим первую связку, для нее возможно $\binom{60}{15}$ вариантов сочетания грибов.
    \item Рассмотрим вторую связку, для нее возможно $\binom{45}{15}$ вариантов сочетания грибов.
    \item Рассмотрим вторую связку, для нее возможно $\binom{30}{15}$ вариантов сочетания грибов.
    \item Рассмотрим вторую связку, для нее возможно $\binom{15}{15}=1$ вариантов сочетания грибов.
    \item Тогда нам необходимы такие варианты где нам необходимо выбрать все связки.
    \item Однако в нашем случае связки считаются упорядоченными т.к мы неявно задаём порядок рассматривая сначала первую связку, затем вторую и т.п
    \item Таким образом $n = \frac{\binom{60}{15} \cdot \binom{45}{15} \cdot \binom{30}{15}}{4!}$
\end{enumerate}


\subsection{Задача 3}
\begin{enumerate}
    \item Между каждой парой выбранных книг должно быть не менее 3х книг. Задача о барьерах.
    \item Для 10 выбранных книг у нас 9 промежутков, каждый занимает минимум 3 места
    \item $3\cdot 9$ книг занято, можем оперировать только $50-3\cdot 9$. Всего 10 интересующих книг. Тогда $\binom{23}{10}$ при этом условие минимум 3 места уже учтено.
\end{enumerate}

\subsection{Задача 4}
\[
\sum_{k=0}^{m} \binom{k}{n}\cdot\binom{m-k}{n} 
\]

\[
\binom{k}{n}\cdot \binom{m-k}{n-k} = \frac{n!}{k!(n-k)!}\cdot \frac{(n-k)!}{(m-k)!(n-m)!}=\frac{n!}{k!(m-k)!(n-m)!} = \frac{n!}{m!(n-m)!}\cdot \frac{m!}{k!(m-k)!}
\]

\[
=\binom{m}{n}\cdot \binom{k}{m}
\]

\[
\sum_{k=0}^{m} \binom{k}{n}\cdot\binom{m-k}{n-k} = \sum_{k=0}^{m} \binom{m}{n}\cdot \binom{k}{m} = \binom{m}{n} \sum_{k=0}^{m} \binom{k}{m} = \binom{m}{n}\cdot 2^m
\]


\subsection{Задача 5}

\[
(\sum_{k=0}^{n} \binom{k}{n})^2 = ((1+1)^n)^2=(1+1)^2n= \sum_{k=0}^{2n} \binom{k}{2n}
\]

\subsection{Задача 6}
\begin{enumerate}
    \item $C(n, k) = \frac{n!}{k!(n-k)!}$
    \item Функция строго возрастает при $n\ge k$
    \item $\forall k C(n,k)=m$ т.е имеет не более одного решения n для произвольного m
    \item Рассмотрим k>m. Минимальное значение C достигается при n=k и равно 1. Однако если мы возьмем n=k+1 то C=k+1. Если k+1>m то для всех $n\ge k+1 C(n, k) \ge k+1 > m$
    \item Таким оразом при $k\ge m $ нет решений
    \item Таким образом следует, что k может принимать значения только от 1 до m-1. Для каждого такого k существует не более одного m удовлетворяющее уравнению $C(n, k)=m$
    \item Следовательно общее кол-во пар $(n, k)$ для которых $C(n, k)=m$ не превышает m-1 что конечно. 
\end{enumerate}

\subsection{Задача 7}
\begin{enumerate}
    \item Воспользуемся теоремой Люка: бином нечетен тогда и только тогда, когда двоичные цифры в двоичном разложении n являются подмножеством битов m
    \item Т.е каждый бит k не превосходит соответствующий бит n в двоичной системе.
    \item если $n=\sum_{i=0}^{m}a_i2^i$ $k=\sum_{i=0}^{m}b_i2^i$ то C(n, k) нечетен тоже самое, что и $\forall i b_i \le a_i$
    \item Если $a_i=1 \Rightarrow b_i = 0 \lor b_i=1$
    \item Если $a_i=0 \Rightarrow b_i=0$
    \item Тогда для каждого $a_i=1$ есть 2 варианта выбора $b_i$, а для $a_i=0$ только один.
    \item Тогда общее количество нечентных $C(n, k)=\prod_{i=0}^{m}(1+a_i)=2^{\text{Кол-во единиц в n}}$
    \item Тогда в нашем случае $2025_{10} = 11111101001_2$
    \item Общее количество нечетных: $2^8=256$
    \item Общее количество четных - $2025+1-256=1770$; +1 потому что строки нумеруются с нуля(n строка содержит n+1 чисел).
\end{enumerate}

\subsection{Задача 8}
\begin{enumerate}
    \item У нас есть n людей k людей имеют 50р, 50 рублей стоимость кино, если дают купюру в 100 рублей - мы даём 50 рублей сдачи т.е минус 50 рублей из кассы.
    \item Далее у нас есть купюра в 100 рублей, если мы разменяли, то это тупо мертвый груз с которым мы ничего не можем делать(100 рублей осталось)
    \item Тогда, согласно условию, у нас есть возможность добавить 50р в кассу, убавить кассу на 50р при этом добавить 100 р. при обоих случаях у нас тратится билет. Всего n билетов
    \item Можем просто убрать 100р, роли не играет. Т.е остались варианты 50 р или -50 рублей сделать, пусть будет +1 и -1 соответственно
    \item Согласно условию количество людей с 100 рублями, не превышает количество с 50 рублями. $k \ge n-k$
    \item Представим координатную ось, где ось ординат - количество 50 рублевых купюр, ось абсцисс - количество людей.
    \item Тогда минимальной допустимой точкой будет $k=n-k \Rightarrow 2k-n=0$, где мы получаем, что каждому 50 рублёвому соответствует 100 рублёвый, чтобы они друг-друга компенсировали и в сумме в кассе было 0.
    \item Пойдём из точки (0, 0) в точку (n, 2k-n), существует способы (+1, -1) (+1, +1) человек с 50 рублями и человек со 100 рублями соответственно.
    \item Всего будет n-k способов использовать -1 и k способов использовать +1, тогда общее количество перемещений - $k+n-k = n $
    \item Тогда общим количеством перемещений будет $\binom{n}{k}$ нужно выбрать k позиций для (+1, +1) и это будет тоже самое что и $\binom{n}{n-k}$
    \item Теперь посчитаем количество плохих траекторий. Для этого мы построим биекцию юмежду плохими траекториями, вудещими из (0, 0) в (n, 2k-n) и произвольными траекториями.
    \item Далее мы отражаем относительно прямой y = -1 т.к в этой точке мы получаем уже достаточное условие того, что траектория становится неправильной (2k-n<0)
    \item Получаем точку (n, -n-k-2) и добраться до нее можно сделав n+1 шаг k способами
    \item Количество хороших траекторий - разница всех и плохих - $(\binom{n}{k}-\binom{n}{k+1})k!(n-k)!$ т.к можно разместить людей еще различно, а не конкретно скобочки

\end{enumerate}

\subsection{Задача 10}
\begin{enumerate}
    \item На любом начальном отрезке строки количество символов ( не меньше, чем количество символов)
    \item Ось ординат - количество открытых скобочек, абсцисс - общее количество скобочек
    \item Будем рассматривать на координатной плоскости из точки (0, 0) в (2n, 0) способами (+1, +1) (+1, -1) если скобка открывающая и закрывающая соответственно.
    \item Весь путь 2n шагов, Если количество открывающих скобок m, то конечная точка будет на высоте k=2m-2n, $k\ge 0 \Rightarrow m\ge n$ 
    \item Для каждого пути, нарушающего условие(опускается ниже оси x), существует соответствующий путь, отраженный относительно уровня -1. Количество таких плохих путей с m открывающими скобками равно $\binom{2n}{m-1}$
    \item $\forall m\ge n$ количество допустимых путей с m открывающими скобками равно $\binom{2n}{m}-\binom{2n}{m-1}$
    \item Суммируем $\sum_{m=n}^{2n} \binom{2n}{m}-\binom{2n}{m-1}=\binom{2n}{2n}-\binom{2n}{n-1}=\binom{2n}{n}$
\end{enumerate}

\subsection{Задача 11}
\begin{enumerate}
    \item Это та же самая задача о количестве правильных скобочных последовательностей. Только теперь нам без разницы, (( или )) у нас т.к это одно и то же
    \item Тогда обозначим за k - красные b - черные
    \item Определим пару убирающейся, если она выглядит следующим образом: $rr\lor bb$
    \item Тогда количество сочетаний тех, которые нас интересуют($rr\lor bb$) равно количеству сочетаний $rb \lor br$ т.к всего можно из двух букв сделать 4 комбинации и они делятся строго на 2 группы.
    \item Это задача на правильную скобочную последовательность. пустая строка - псп. Если строка S это ПСП - rSb или bSr это также ПСП. если S и T это ПСП, то их конкатенация также ПСП.
    \item Тогда воспользуемся числом Каталана и получим, что если цвет был одним и тем же, то $\frac{1}{27}\binom{26}{52}$
    \item Тогда для двух цветов $\frac{1}{27}\binom{26}{52}^2$
\end{enumerate}

\end{document}