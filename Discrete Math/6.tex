\documentclass[a4paper,12pt]{article}

% Кодировка и язык
\usepackage[utf8]{inputenc}
\usepackage[russian]{babel}

% Математические пакеты
\usepackage{amsmath,amsfonts,amssymb}

% Графика
\usepackage{graphicx}
\usepackage{tikz}
\usetikzlibrary{shapes.geometric, calc}
\usepackage{pgfplots}
\pgfplotsset{compat=1.18} % Добавлено для устранения предупреждения PGFPlots

% Геометрия страницы
\usepackage{geometry}
\geometry{top=2cm, bottom=2cm, left=2.5cm, right=2.5cm}

% Гиперссылки
\usepackage{hyperref}

% Плавающие объекты
\usepackage{float}

% Дополнительные пакеты
\usepackage{venndiagram}

% Настройки заголовка
\title{Домашнее задание}
\author{Студент: \textbf{Ростислав Лохов}}
\date{\today}

\begin{document}

% Титульный лист
\begin{titlepage}
    \centering
    \vspace*{1cm}

    \Huge
    \textbf{Домашнее задание}

    \vspace{0.5cm}
    \LARGE
    По курсу: \textbf{Дискретная математика}

    \vspace{1.5cm}

    \textbf{Студент: Ростислав Лохов}

    \vfill

    \Large
    АНО ВО Центральный университет\\
    \vspace{0.3cm}
    \today

\end{titlepage}

% Содержание
\tableofcontents
\newpage

% Основной текст
\section{Индукция и реккурентные уравнения}


\subsection{Задача 1}
\begin{enumerate}
    \item База индукции: Рассмотрим круг из 1 красного шарика и одного синего, тогда можно составить 1 круг беря попарно шарики. Также взяв всего 2 шарика можно составить только один круг(КС)
    \item Индукционный переход: Допустим можно получить  любую комбинацию из k красных и k синих шаров в круге, прибавляя пары различных цветов.
    \item Рассмотрим k+1 красных и столько же синих шаров. Тогда т.к шариков одинаковое количество, то существует пара с различными цветами(КС) или (CК). Уберем ее, получим, что осталось k красных и k синих шариков.
    \item По индукционному предположению, мы можем из k красных и k синих шаров получить любой круг, прибавляя k пар соседних разноцветных шариков
    \item Далее после k операций добавления пар мы добавляем убранную пару разноцветных шариков на место где она была расположена в исходной комбинации, мы получим исходную комбинациюю из k+1 красных и столько же синих шаров.
    \item Таким образом на основании принципа математической индукции любая комбинация из n красных и синих шаров расположенных по кругу, может быть получена последовательно добавляя пары из соседних разноцветных шариков в изначально пустой круг
\end{enumerate}

\subsection{Задача 2}
\begin{enumerate}
    \item База индукции: Рассмотрим две ступеньки, человек стоит на одной из ступенек
    \item Всего существует 4 возможных варианта комбинаций указателй на 2х ступеньках
    \item (UU) (UD) (DU) (DD) - UP, Down, справа - верх лестницы, слева - низ.
    \item Тогда если он стоит на любой из ступенек у крайних двух ((UU), (DD)), то очевидно, что он сойдет со ступенек
    \item (UD) - Если начинаем на U, то мы поднимаемся в D, далее U сменилась на D, после D мы уходим на D и далее уходим с лестницы(Эффект как D), Если начинаем с D, будет эффект U
    \item (DU) - Если начинаем на D, то ведёт себя как D, Если U - то как U
    \item Тогда из двух произвольных ступенек, мы получаем либо эффект UU либо DD
    \item База доказана
    \item Индукционный переход: Допустим мы можем уйти с лестницы из k ступенек.
    \item Рассмотрим k+1 ступенек, Возможны два варианта - лестница полностью состоит из одного и того же знака, в таком случае мы просто уходим с лестницы, или на лестнице есть хотя бы два различных знака.
    \item Тогда эффект хотя бы одной такой пары превращается в U или D, из прошлых рассуждений. В таком случае общее результирующее будет k, и мы приходим к индукционному предположению
    \item Значит таким образом  рано или поздно человек покинет лестницу.
\end{enumerate}

\subsection{Задача 3}
\begin{enumerate}
    \item Есть множество $S = \{1, 2, 3 ,..., 2^n\}$ необходимо найти количество способов раскрасить каждое число в 2 цвета, но при этом, если два числа образуют сумму некоторого числа, которое является степенью двойки.
    \item Тогда пусть есть ребро между двумя вершинами, если сумма этих чисел даёт степень двойки.
    \item $a+b \in S, a \ne b$ $a+b \in \{3; 2^n + 2^{n+1}-1\}$ - диапазон суммы. Единственные натуральные степени двойки имеют диапазон от 2 в степени 2 до 2 в степени n+1
    \item Заметим, что $2^n= 2^{n-1}+ 2^{n-1}$ т.к числа должны быть различными, эта запись не  приводит к ребру, следовательно $2^{n-1}$ не используется при создании ребер.
    \item Общее количество компонент -C(n) = n+1. Докажем по индукции.
    \item База: множество $S=\{1, 2\}$ - каждая пара не даёт степень двойки, поэтому рёбер нет. Следовательно каждая вершина является компонентой, т.е общее количество - 2. C(1) = 2
    \item Предположим, что для $S_{n}=\{1, 2, ... 2^{n}\}$ количество компонент $C(n) = n+1$
    \item Рассмотрим множество $S_{n+1}=\{1, 2, ... 2^{n+1}\}$
    \item Разобьём на 2 множества: $A=\{1,2,3...2^n\}, \quad B=\{2^n+1, 2^n+2 ... 2^{n+1}\}$
    \item Тогда A совпадает с предположением и имеет n+1 компонент.
    \item Рассмотрим B, заметим, что если взять любые два числа, то их сумма всегда будет больше $2^{n+1}$
    \item Таким образом в B между любыми двумя вершинами ребер нет, т.е все вершины изолированы. Тогда число вершин в B = $2^n$
    \item Теперь нам необходимо понять какие рёбра соединяют А и В.
    \item Заметим, что там необходимо найти такие рёбра $(a, 2^{n+1}-a)$  (из уравнения $a+b=2^{n+1}$)
    \item Т.е для каждого а из А соответсвуеющая вершина $2^{n+1}-а$ из В присоединяется к компоненте, в которой находится а. Каждый такой переход обьединяет две компоненты (одну из А и одну из В) в одну.
    \item Изначально до связывания А и В обратно у нас получилось $n+1+2^n$ компонент
    \item Берем из А все а от $1$ до $2^n-1$ т.е каждое такое ребро будет соединять а с изолированной на множестве В вершину. Таких ребер $2^n-1$ и каждое уменьшает общее число ребер на 1 т.к мы считаем дважды.
    \item Тогда $C(n+1) = n+1+2^n-2^n+1=n+2$
    \item Т.е всего для $S = \{1, 2, 3 ,..., 2^n\}$ n+1 компонент.
    \item Далее надо доказать, что каждая компонента имеет всего 2 варианта раскраса.
    \item Т.к каждая компонента представляет собой цепочку или одну вершину, то у каждой такой компоненты мы можем выбрать всего 2 раскраса(определение цветов первым + чередование) (цепочка потому что иначе мы бы не смогли покрасить так, чтобы любые две соседствующих вершины были разных цветов при 2х цветах максимум)
    \item Тактим образом каждая компонента имеет два цвета, а значит всего способов раскрасить числа в 2 цвета - $2^{n+1}$
\end{enumerate}

\subsection{Задача 4}

\subsection{Задача 5}

\subsection{Задача 6}
\begin{enumerate}
    \item Одна окружность делит на 2 плоскости, т.е $R(1)=2$
    \item Предположим, что $R(k)=k^2-k+2$, $R(1) = 2$ - база доказана.
    \item Предположим, что формула верна для некоторого k.
    \item Тогда $R(k+1)=(k+1)^2-(k+1)+2$
    \item По условию k+1 окружность пересекается с каждой из к уже проведенных окружностей ровно в 2х точках
    \item Эти 2k точек разбивают новую окружность на 2k дуг, каждая дуга проходит через одну из областей, которые уже образованы k окружностями
    \item При добавлении дуги, область разделяется на две. Следовательно каждая из 2k дуг добавляет по одной области, т.е новых обласей ровно 2k
    \item Тогда $R(k+1)=R(k)+2k$
    \item $R(k+1)=(k^2-k+2)+2k=k^2+k+2=(k+1)^2-(k+1)+2$
    \item Таким образом переход индукции выполнен, а значит для любого натурального n, число областей, на которые n окружностей делят плоскость равно $n^2-n+2$
\end{enumerate}

\subsection{Задача 7}
\begin{enumerate}
    \item После четного числа прыжков лягушка может находится только в вершинах A C E.
    \item Обозначим $a_k, c_k, e_k$ число путей длины 2k из А в А, С и Е.
    \item В силу симметрии $C_k=E_k$:
    \item $C_{k+1} = A_k + 3C_k$
    \item $A_{k+1} = 2A_k+2C_k$
    \item $С_{k+2}=A_{k+1}+3C_{k+1}=2A_{k}+2C_{k}+3C_{k+1}=2(C_{k+1}-3C_k)+2C_k+3C_{k+1}=5C_{k+1}-4C_k$
    \item $\lambda^2=5 \lambda -4$ $\lambda = 4, \quad \lambda = 1$
    \item $C_n = C_1 4^n+C_2$ $C_1 = \frac{1}{3}, \quad C_2 = -\frac{1}{3}$
    \item $C_n = \frac{1}{3}(4^n-1)$
    \item Обозначим через $B_k$ число путей длины $2k-1$, ведущих из A в B, тогда
    \item $B_{k+1} = 3b_k$ т.к за два прыжка можно двумя способами вернуться из B в B и одним способом попасть из B в F.
    \item Т.к $С_k = B_k$, то $C_{k+1}=3C_k, k>0, C_1=1 \Rightarrow C_k=3^{k-1}$
\end{enumerate}

\subsection{Задача 8}
\begin{enumerate}
    \item Пусть f(n) - число слов длины n c четным числом букв А. Тогда g(n) - с нечетным А.
    \item Если слово длины n-1 имеет четное число букв А, то при добавлении не А, четность не меняется получаем 2f(n-1) вариантов.
    \item Если же добавляем А, то будет нечет число А, f(n-1)
    \item Если слово такой же длины имеет нечетное число А, то при  добавлении не А, четность не меняется и остаётся 2g(n-1).
    \item Если А, то g(n-1) вариант.
    \item $f(n)=2f(n-1)+g(n-1) \land g(n)=f(n-1)+2g(n-1) \land f(0)=1 \land f(0)=1 \land f(n)+g(n)=3^n$
    \item $f(n) = 2f(n-1) + \Bigl(3^{n-1} - f(n-1)\Bigr) = f(n-1) + 3^{n-1} =  1 + \sum_{k=0}^{n-1} 3^k = \frac{3^n+1}{2}$ сумма геом прогрессии
\end{enumerate}


\subsection{Задача 9}

\subsection{Задача 10}
\begin{enumerate}
    \item Башню 2 x 2 x 1 можно составить 2мя способами(поворот на 0 градусов относительно горизонтальной плоскости и на 90 градусов)
    \item Далее башня 2 x 2 x 2  4 варианта, если кирпичи лежат большей стороной к основанию и 5 вариантов, если вертикально стоят хотя бы 2.
    \item Т.е у нас два состояния для предыдущей башни + 5 вариантов после наклона.
    \item Тогда $A_n=A_{n-1}+A_{n-2}$ т.е текущий слой это сумма состояний предыдущего + сумма состояний предыдущего в обьединении с пред предыдущим.
    \item $A_1 = 2, A_2 = 9$
    \item $\lambda_{1,2}=\frac{1\pm\sqrt{5}}{2}$
    \item $A_n = \frac{7\sqrt{5}-3}{2\sqrt{5}}\left(\frac{1+\sqrt{5}}{2}\right)^n + \frac{7\sqrt{5}+3}{2\sqrt{5}}\left(\frac{1-\sqrt{5}}{2}\right)^n$
\end{enumerate}

\subsection{Задача 11}
\begin{enumerate}
    \item На каждом шаге у деда есть выбор, вложить в мешок мешок или засунуть подарок, в каждом мешке либо два подарка, либо два мешка. Надо разложить все подарки.
    \item Заметим, что если мы положим 2 подарка в один мешок, то далее мы не сможем ложить подарки
    \item Тогда у нас на каждом этапе будет выбор, или два мешка, или мешок и подарок, первое бессмысленно, поскольку мы не тратим подарки, а второе имеет смысл, тогда реккурент задан следующим образом
    \item $f(n)=\binom{n}{2}f(n-1)$ т.к подарки различны
    \item $f(n)=\prod_{k=2}^{n}\frac{k(k-1)}{2}f(1)$
    \item $f(n)=\frac{n!(n-1)!}{2^{n-1}}$
\end{enumerate}

\end{document}
