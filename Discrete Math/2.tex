\documentclass[a4paper,12pt]{article}

% Кодировка и язык
\usepackage[utf8]{inputenc}
\usepackage[russian]{babel}

% Математические пакеты
\usepackage{amsmath,amsfonts,amssymb}

% Графика
\usepackage{graphicx}
\usepackage{tikz}
\usetikzlibrary{shapes.geometric, calc}
\usepackage{pgfplots}
\pgfplotsset{compat=1.18} % Добавлено для устранения предупреждения PGFPlots

% Геометрия страницы
\usepackage{geometry}
\geometry{top=2cm, bottom=2cm, left=2.5cm, right=2.5cm}

% Гиперссылки
\usepackage{hyperref}

% Плавающие объекты
\usepackage{float}

% Дополнительные пакеты
\usepackage{venndiagram}

% Настройки заголовка
\title{Домашнее задание}
\author{Студент: \textbf{Ростислав Лохов}}
\date{\today}

\begin{document}

% Титульный лист
\begin{titlepage}
    \centering
    \vspace*{1cm}

    \Huge
    \textbf{Домашнее задание}

    \vspace{0.5cm}
    \LARGE
    По курсу: \textbf{Дискретная математика}

    \vspace{1.5cm}

    \textbf{Студент: Ростислав Лохов}

    \vfill

    \Large
    АНО ВО Центральный университет\\
    \vspace{0.3cm}
    \today

\end{titlepage}

% Содержание
\tableofcontents
\newpage

% Основной текст
\section{Предикаты и кванторы. Логика первого порядка}

ДИСКЛЕЙМЕР: Мне очень нужно попасть на красно-черный уровень, для этого нужно 8  баллов минимум набрать в течение 3х недель, не придирайтесь строго пожалуйста:) Я попал сюда потому что пропустил тест:(

\subsection{Задача 1}
\begin{enumerate}
    \item $y=2x$ - верно, для любого натурального x существует натуральный y, что выполняется
    \item $y=2x$ - неверно, например $y=7 \Leftarrow x=3.5$ что не является натуральным
    \item $x=2y$ - неверно, также $x=7 \Leftarrow y=3.5$ что не является натуральным
\end{enumerate}

\subsection{Задача 2}
\begin{enumerate}
    \item $\forall z(z > x \to z > y)$
    \item $\forall a((\exists x (ax^2+4x-2=0))\leftrightarrow (a \ge -2))$
    \item $\forall x((x^3-3x<3) \to (x < 10))$
    \item $((\exists x(x^2+5x=w))\land(\exists y (4-y^2=w))) \to (-10 \le w \le 10)$
\end{enumerate}

\subsection{Задча 3}
\begin{enumerate}
    \item S(x) - дискретная математика это просто, U(x) - что-то в математике понимает, тогда $\forall x(S(x) \to \overline{U(x)})$ Если Влад прав, то отрицание этого утверждения. $\forall x(S(x) \land U(x))$
    \item Может, но не обязано
    \item Может, но не обязано
    \item Не обязано и не может
    \item Не обязано и не может
    \item Может и обязано
    \item Может и обязано
    \item Может и не обязано
    \item Не обязано и не может
    \item Не обязано и не может
\end{enumerate}

\subsection{Задача 4}
\begin{enumerate}
    \item х является простым числом, множество истинности - множество простых чисел.
    \item х является наименьшим общим кратным чисел y и z. Множество истинности: ${x \in N| x = lcm(y, z)}={lcm(y, z)}$ lcm - least common divisor
    \item Утверждение верно в поле вещественных чисел. Множество истинности - верно всегда
    \item Для любого вещественного числа х существует ровно два различных вещественных квадратных корня. Множество истинности - верно всегда.
\end{enumerate}

\subsection{Задача 5}
\begin{enumerate}
    \item $\forall B S(A,B)$ - множество является пустым тогда и только тогда, когда оно является подмножеством любого множества
    \item $\forall B S(B, A)$ - прикол в счётности, а точнее несчётности)
    \item $S(B, A) = S(A, B)$
    \item \begin{align*}
        \exists X \big( 
            & \neg \big( \forall Z \, S(X, Z) \big) \land S(X, A) \land \\
            & \forall Y \big( S(Y, A) \land \neg \big( \forall Z \, S(Y, Z) \big) \rightarrow \\
            & \quad \big( S(Y, X) \land S(X, Y) \big) \big) 
        \big)
        \end{align*}
    \item \begin{align*}
        \exists X \exists Y \big( 
            & \text{одноэлементное}(X) \land 
              \text{одноэлементное}(Y) \land \\
            & \neg ( S(X, Y) \land S(Y, X) ) \land S(X, A) \land S(Y, A) \land \\
            & \forall Z \big( S(Z, A) \land \text{одноэлементное}(Z) \rightarrow \\
            & \quad ((S(Z, X) \land S(X, Z)) \lor (S(Z, Y) \land S(Y, Z))) \big) 
        \big)
        \end{align*}
\end{enumerate}

\subsection{Задача 6}
\begin{enumerate}
    \item $\overline{O} \to \overline{Д}$
    \item Эквивалентно т.к контрапозиция
    \item Обратное к исходному. Из того что кто-то ничего не делает, не следует, что он единственный, кто не ошибается.
    \item Эквивалентно
    \item не Эквивалентно, из того, что кто-то ошибается, не следует, что он обязательно что-то делает.
    \item логически Эквивалентно, обращение исходного
    \item логически Эквивалентно
    \item логически Эквивалентно $\overline{Д \land \overline{O}} \to \overline{Д} \lor O \to Д \to O$
    \item логически Эквивалентно $\overline{Д} \lor O \to Д \to O$
    \item 1 - авежзи 2- бгд
\end{enumerate}

\subsection{Задача 7}
\begin{enumerate}
    \item Доказательство пытается вывести значение S, но при этом на шаге 2 мы уже используем S для доказательства для других треугольников. Типо x = 2x - x решить.
\end{enumerate}
\end{document}