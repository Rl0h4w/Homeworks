\documentclass[a4paper,12pt]{article}

% Кодировка и язык
\usepackage[utf8]{inputenc}
\usepackage[russian]{babel}

% Математические пакеты
\usepackage{amsmath,amsfonts,amssymb}

% Графика
\usepackage{graphicx}
\usepackage{tikz}
\usetikzlibrary{shapes.geometric, calc}
\usepackage{pgfplots}
\pgfplotsset{compat=1.18} % Добавлено для устранения предупреждения PGFPlots

% Геометрия страницы
\usepackage{geometry}
\geometry{top=2cm, bottom=2cm, left=2.5cm, right=2.5cm}

% Гиперссылки
\usepackage{hyperref}

% Плавающие объекты
\usepackage{float}

% Дополнительные пакеты
\usepackage{venndiagram}

% Настройки заголовка
\title{Домашнее задание}
\author{Студент: \textbf{Ростислав Лохов}}
\date{\today}

\begin{document}

% Титульный лист
\begin{titlepage}
    \centering
    \vspace*{1cm}

    \Huge
    \textbf{Домашнее задание}

    \vspace{0.5cm}
    \LARGE
    По курсу: \textbf{Дискретная математика}

    \vspace{1.5cm}

    \textbf{Студент: Ростислав Лохов}

    \vfill

    \Large
    АНО ВО Центральный университет\\
    \vspace{0.3cm}
    \today

\end{titlepage}

% Содержание
\tableofcontents
\newpage

% Основной текст
\section{Основы теории графов}

\subsection{Задача 1}
\begin{enumerate}
    \item Вершины называются смежными если между ними есть ребро
    \item a) Для любых двух смежных вершин есть ровно одна вершина смежная с ними обеими. 
    \item Т.е $K_3$ граф. т.е наборы треугольников которые не связаны между собой. 
    \item Если у нас n таких подграфов, то общее кол-во ребер 3n. 
    \item Т.к 100 не делится на 3, то граф удовлетворяющий условию не может иметь ровно 100 ребер
    \item б) для любых двух смежных вершин есть ровно две вершины смежные с ними обеими. Т.е $K_4$ граф.
    \item $K_4$ граф для любых двух смежных вершины содержит ровно 2 вершины смежные с этими двумя.
    \item Кол-во ребер в $K_4$ графе $\binom{4}{2}=6$ ребер. Если наш граф, содержит такие подграфы в кол-ве n, то общее кол-во ребер 6n, что не является делителем 100. Следовательно нет
\end{enumerate}

\subsection{Задача 2}
\begin{enumerate}
    \item Пусть k = 1. Тогда из одного любого города можно попасть в любой другой город, причем только 1. Т.е будет набор изолированных городов или пар связанных городов  или в лучшем случае одну линию.
    \item Если есть изолированные города - условие связанности не выполнено
    \item Если есть пары допустим A-B, C-D - условие связанности также не выполняется. Мы не можем попасть из B в C
    \item Если есть линия - пусть будет A-B-C..-H то из А в H попасть не более чем через один город нельзя
    \item Пусть k=2. 
    \item Тогда возможны случаи, когда они собираются поотдельности. Вдвоем, и наконец в треугольники.
    \item Как мы сказали ранее, изолированная вершина нас не интересует. Вершины связанные друг с другом т.е степень 1 также нас не интересуют.
    \item Чтобы связать все 8 городов и чтобы каждая вершина была степени 2 нужен цикл. Т.е A-B-C-D-E-F-G-H-A. Любая друга структура будет либо несвязной. Либо будет иметь вершины степени 1.(линия) что не меняет картины.
    \item Рассмотрим наш цикл A-B-C-D-E-F-G-H-A.
    \item Допустим возьмем за опорный - город А.
    \item Тогда города на расстоянии 1 это B и H
    \item На расстоянии 2 - C и G
    \item На расстоянии 3 - D и F
    \item Тогда мы не можем добраться из А до города D и F, т.к оно противоречит условию, что максимум мы обходим 1 город.
    \item Таким образом k=2  также не подходит
    \item k=3
    \item Построим пример: соединим ребра по кругу, как при k=2. Далее добавим диагонали, соединяющие противоположные города.
    \item Доказательство того, что мы можем попасть в любую вершину.
    \item Пусть мы взяли вершину i. все вершины пронумерованы от 0 до 7 включительно
    \item i-1 i+1 достижимы за 1 ход
    \item i+4 i-4 достижимы за 1 ход
    \item i-2, i+2 достижимы за 2 хода $(i+1->i+2 \land i-1 -> i-2)$
    \item i-3, i+3 достижимы за 2 хода $(i-1 -> i+3 \land i+1 ->i-3)$
    \item Таким образом k=3
\end{enumerate}


\subsection{Задача 4}
\begin{enumerate}
    \item Рассмотрим одного из 50 человек. Назовём его A. 
    \item Оставшиеся 49 человек можно разделить на 2 группы: N и M - множество людей знакомых и не знакомых с А соответственно.
    \item Возможны 2 случая - мощность N четна и нечетна.
    \item Если N нечетно, в таком случае найдется человек B в группе N, у которого четное число знакомых среди людей в N.
    \item Теперь рассмотрим пару (A, B). Их общие знакомые - среди оставшихся 48 человек - люди из N, которые знакомы с B. Поскольку B имеет чётное число знакомых в N, то число общих знакомых у А и B среди остальных 48 человек чётно. Следовательно А-В искомая пара.
    \item Если N четно, то M нечетно.
    \item Тогда в группе M незнакомых с А нечетное число людей, следовательно в М найдется такой человек С, у которого четное число знакомых в М. Поскольку С незнаком с А, а всего у него четное число знакомых, то у него четное число знакомых в N. C-A - искомая пара.
\end{enumerate}

\subsection{Задача 5}
\begin{enumerate}
    \item Девочка может быть знакома с любым количеством мальчиков. Т.к девочек n+1 и вариантов n+1 (добавляется то, что девочка незнакома с мальчиками)
    \item Общее число пар знакомых - сумма всех степеней девочек. т.е $\frac{n(n+1)}{2}$
    \item Каждый мальчик знаком с одним и тем же числом девочек. Пусть будет d
    \item Общее кол-во знакомств - n*d
    \item $nd=\frac{n(n+1)}{2}\Rightarrow d = \frac{n+1}{2}$ - кол-во знакомых у каждого мальчика.
    \item Тогда d должно быть целым, а значит n+1 должно быть четно, значит n нечетно.
    \item Пусть $k \in \{0, 1, \dots, m\}$, $n = 2m + 1$, где $m \in \mathbb{N}$. Разобьем девочек на пары, всего $m+1$ пара. Первую девочку $k$-й пары познакомим с $k$ мальчиками, вторую - с остальными $n-k$ мальчиками. При этом каждый мальчик знаком с ровно одной девочкой из каждой пары.
\end{enumerate}

\subsection{Задача 6}
\begin{enumerate}
    \item Возьмем полный граф 64 вершины. Тогда кол-во ребер $\binom{64}{2}=2016$ 
    \item Мы не можем составить простой путь в полном графе из 64 вершин - длина такого пути будет 63 - иначе, если больше 63 ребер, то будет не простой путь.
    \item Осталось 9 ребер и 936 вершин. Пусть просто какие то из них связаны в цепочку.
    \item Итого 10 вершин связаны в цепочку - первый граф. Второй граф - 926 изолированных вершин. Третий граф - полносвязный граф на 64 вершины.
    \item Таким образом никакой из этих графов не будте связан с другим и никакой из них не будет иметь путь длины 64.
\end{enumerate}

\subsection{Задача 7}
\begin{enumerate}
    \item Всего ребер: $400*201 = 2E \Rightarrow E=40200$ 
    \item Для двудольного графа с 400 вершинами максимальное кол-во ребер - $400^2/4 = 40000$
    \item А значит т.к наш граф имеет большее кол-во ребер, чем двудольный граф(граф который не содержит циклов длины 3) то наш граф обязательно содержит цикл длины 3.
\end{enumerate}




\end{document}
