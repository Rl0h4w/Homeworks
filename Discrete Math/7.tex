\documentclass[a4paper,12pt]{article}

% Кодировка и язык
\usepackage[utf8]{inputenc}
\usepackage[russian]{babel}

% Математические пакеты
\usepackage{amsmath,amsfonts,amssymb}

% Графика
\usepackage{graphicx}
\usepackage{tikz}
\usetikzlibrary{shapes.geometric, calc}
\usepackage{pgfplots}
\pgfplotsset{compat=1.18} % Добавлено для устранения предупреждения PGFPlots

% Геометрия страницы
\usepackage{geometry}
\geometry{top=2cm, bottom=2cm, left=2.5cm, right=2.5cm}

% Гиперссылки
\usepackage{hyperref}

% Плавающие объекты
\usepackage{float}

% Дополнительные пакеты
\usepackage{venndiagram}

% Настройки заголовка
\title{Домашнее задание}
\author{Студент: \textbf{Ростислав Лохов}}
\date{\today}

\begin{document}

% Титульный лист
\begin{titlepage}
    \centering
    \vspace*{1cm}

    \Huge
    \textbf{Домашнее задание}

    \vspace{0.5cm}
    \LARGE
    По курсу: \textbf{Дискретная математика}

    \vspace{1.5cm}

    \textbf{Студент: Ростислав Лохов}

    \vfill

    \Large
    АНО ВО Центральный университет\\
    \vspace{0.3cm}
    \today

\end{titlepage}

% Содержание
\tableofcontents
\newpage

% Основной текст
\section{Индукция и реккурентные уравнения}


\subsection{Задача 1}
\begin{enumerate}
    \item Всего вариантов сочетаний с повторениями: $\binom{12}{8} = 495$
    \item $1, 1, 2... ...$ - для последнего можем выбрать лексикографически меньшее, будет 
    \item $1, 1, 1, x_4, x_5, x_6, x_7, x_8$ каждый такой x от 1 до 5 включительно, значит всего вариантов $\binom{9}{5} = 126$
    \item Далее $1, 1, 2, 2, 2, x_6, x_7, x_8$ - $x_6$ может быть 2 или 3, т.к иначе мы бы нарушали условие возрастания.
    \item Если 6 элемент равен 2, то оставшиеся 2 элемента принимают от 2 до 5 включительно. Если 3м - то от 3х до 5 включительно.
    \item Значит число сочетаний с повторениями = 10 и 6 соответственно
    \item Таким образом 142 комбинации, значит номер заданного сочетания 143
    \item Если первый элемент 1, то получим, что каждый из оставшихся 7 элементов принимает значение от 1 до 5 включительно. Значит всего $\binom{11}{7}=330$, 127 меньше 495, значит продолжаем
    \item Значит первая цифра 1
    \item Далее второй элемент будет 1, т.к минимальное значение интересует, значит каждое из 6 оставшихся принимает значение от 1 до 5 включительно. Всего $\binom{10}{6}=210$ 127 меньшне чем 210. Продолжаем
    \item Предположим, что и дальше будет элемент 1, значит оставшиеся 5 принимают значение от 1 до 5 включительно. Значит на них приходится $\binom{9}{5}=126$, что меньше 127. т.к мы вышли за пределы нумерации, значит тут двойка.
    \item Если двойка, значит оставшиеся 5 чисел принимают значения от 2 до 5, т.е $\binom{8}{5}=56$
    \item Таким образом ответ (1, 1, 2, 2, 2, 2, 2, 2).
\end{enumerate}

\subsection{Задача 2}

\subsection{Задача 3}
\begin{enumerate}
    \item Разобьем на 2 кучи, по 50 монет, пусть будет куча А и Б
    \item Если куча А больше Б. Рассмотрим подкучи кучи Б 25 и 25, С и Д соответственно
    \item Если куча С больше или меньше кучи Д, т.е неравна. то 50 монет кучи А настоящие т.к 50 монет больше чем 49 + 1 из кучи Б. Т.е в куче Б содержится монета которая легче. Т.е 49+1 фальшивая монета легче чем 50 настоящих.
    \item В случае если куча С и Д равны, то фальшивая монета тяжелее, т.к куча А > Б и подкучи C и Д равны. Т.е подкуча А больше при 50 монетах чем куча Б при куче в 50 монет. Таким образом фальшивая монета тяжелее.
    \item За одно нельзя т.к мы получим неравенство, а следовательно, мы не сможем знать в какой куче всё нормально, а в какой нет. Следовательно только за 2 взвешивания
\end{enumerate}


\subsection{Задача 4}
    \begin{enumerate}
        \item Пусть наше число закодировано в двоичном виде.
        \item Тогда пусть каждый iый кролик пьет вино если i ый бит равен 1.
        \item Тогда на следующий день мы знаем какие кролики умерли. И номера этих кроликов указывают на биты равные 1 в двоичном представлении номера бутылки, которое отравлено.
        \item Так как кролики пили одновременно, то вышел 1 день.
        \item Если кроликов 5
        \item Допустим можно за 3 дня. Каждый кролик может. Пить в день 1. В день 2. В день 3. Ну или быть мертвым в один из дней.
        \item Присвоим каждой бутылке уникальный номер из вектора в $(k_1, k_2, k_3, k_4, k_5), k \in {0, 1, 2, 3}$
        \item Кролик i пьет из бутылка в день $k_i$
        \item Тогда для каждого кролика Если умер в день d, то $k_i = d$
        \item Если выжил, то $k_i = 0$
        \item Количество комбинаций даёт $(3+1)^5=1024$ состояния различных.
        \item Докажем, что меньше нельзя.
        \item Предположим что можно за 2. Тогда у кролика 3 состояния. Умер в день 1. Умер в день 2 и выжил.
        \item Для 5 кроликов кол-во состояний $3^5=243$ т.е такими способами можно закодировать бутылки.
        \item Следовательно недостаточно двух дней.
        \item Далее предположим, что всего 3 дня. Т.е 4 состояния у кролика. Умер в день 1. Умер в день 2. Умер в день 3. Жив.
        \item Тогда информации $2^10=1024$ что хватает для кодирования 1000 бутылок.
        \item Представим номер бутылки от 0 до 999 в четверичной системе счисления в виде 5-значного числа $(b_1, b_2, b_3, b_4, b_5)_4$, где $b_i \in \{0, 1, 2, 3\}$. Кролик $i$ пьет из этой бутылки в день $b_i$ (если $b_i \in \{1, 2, 3\}$; если $b_i = 0$, то не пьет). Если бутылка отравлена и $b_i = j \in \{1, 2, 3\}$, то кролик $i$ умирает в день $j+1$. На четвертый день мы будем знать, кто умер и в какой день, и сможем определить все $b_i$, а значит, и номер бутылки.
    \end{enumerate}


\subsection{Задача 5}
\begin{enumerate}
    \item Проведём соревнование в 5 группах по 5 машин. Будет 5 быстрейших среди групп. +5 заездов.
    \item Далее мы проводим заезд среди 5 этих сильнейших для определения чемпиона.
    \item Далее у нас есть чемпион и нам надо определить рангом ниже чемпиона. Т.е взять 4 из 5 быстрейших и взять второго из начального заезда с чемпионом. Почему не 3его - он уже ниже нашего вторго и не имеет смысла брать топ 3 игрока. А топ 2 среди группы может обойти остальных.
    \item Таким образом мы определили второго по скорости. Т.е всего $5+1+1=7 $ заездов
    \item Докажем что нельзя меньше 7.
    \item Всего мы можем сравнить 5 машин между собой. Тогда сравним группы по 5. Затем еще этих 5 победителей по 5. Допустим мы хотим знать топ 2. Однако топ 2 среди последней группы может проиграть топ 2 среди группы победителя в первой сетке. Тогда возникает противоречие.
\end{enumerate}

\subsection{Задача 6}

\begin{enumerate}
    \item Разобьем нашу кучу шаров на 3 подкучи. 86 86 85. Будем рассматривать суммы двух куч из этих трёх. 3 запроса для нахождения кучи с 2мя радиоактивными шарами. Осталось 2/3 кучи. Или в худшем случае 172 шара. Если будем суммировать по 1, то мы не сможем сказать т.к множества не будут пересекаться и элементы могут быть просто в разных кучах.
    \item Посмотрим, что будет если разбить на 4 подкучи. 65 64 64 64. Если будем суммировать по 3, то получим 3 проверки и 3/4 кучи. Если будем суммировать по 2. То нам понадобится 7 проверок. Что хуже, учитывая что мы получаем половину, вместо того, чтобы за то же количество -1 поверок получить 4/9 от изначальной кучи в первом варианте.
    \item Для 5 чуть более сложно. Но суть та же, мы не получаем выигрыша в разбиениях и уж тем более в изменении суммирования.
    \item Тогда мы делаем 3 проверки и получаем 2/3 кучи. Проделаем это выбирая наибольшую сумму всегда
    \item 86, 86, 85: 172
    \item 57, 57, 58: 115 
    \item 38, 38, 39: 77
    \item 26, 26, 25: 52
    \item 17, 17, 18: 35
    \item 12, 12, 11: 24
    \item 8, 8, 8: 16
    \item 5, 5, 6: 11
    \item 4, 4, 3: 8
    \item 3, 3, 2: 5
    \item 2, 2, 1: 4
    \item 1, 1, 2 - тут 6 сравнений будет 
    \item Итого 39 сравнений.
    \item Таким образом меняя разбиение и меняя суммирование, выигрыша мы не получим. И наиболее оптимальным поиском будет наш алгоритм который приведет за 39 ходов
\end{enumerate}


\subsection{Задача 8}
\begin{enumerate}
    \item Закодируем в троичную систему счисления.
    \item Для первого взвешивания положим на одну чашу весов те монеты, у которых старший равзряд равен 0. На другую - у которых он равен 2.
    \item Если перетянет чаша с 0, то запишем на бумажке цифру 0. Если 2 - 2. если равновесие - 1.
    \item Для второго взвешивания - также второй разряд.
    \item Для третьего - третий. и так 5 раз.
    \item Мы получили цифру - пятизначное число. Далее определяем фальшивую монету по следующему алгоритму.
    \item Если число совпадает с номером какой-то монеты, то эта монета фальшивая и тяжелее остальных. Если нет - заменим в этом числе все нули на двойки, а все двойки на нули. ТОгда эта монета легче остальных.
\end{enumerate}


\subsection{Задача 9}

\begin{enumerate}
    \item Если бы Илья был бы честным, то всего кол-во вариантов закодировать числа бинарными вопросами было бы $2^n$
    \item Т.к илья может соврать не более одного раза, то для каждого числа существует k+1 возможных последовательностей ответов
    \item Чтобы Алла могла гарантированно определить число, все $200(k+1)$ сценариев должны соотвествовать уникальным последовательностям длины k.
    \item Т.е $200(k+2)\le 2^k$ Наименьшее такое целое $k=12 $
    \item Таким образом за 12 вопросов.
\end{enumerate}


\end{document}
