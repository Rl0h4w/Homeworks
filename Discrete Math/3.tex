\documentclass[a4paper,12pt]{article}

% Кодировка и язык
\usepackage[utf8]{inputenc}
\usepackage[russian]{babel}

% Математические пакеты
\usepackage{amsmath,amsfonts,amssymb}

% Графика
\usepackage{graphicx}
\usepackage{tikz}
\usetikzlibrary{shapes.geometric, calc}
\usepackage{pgfplots}
\pgfplotsset{compat=1.18} % Добавлено для устранения предупреждения PGFPlots

% Геометрия страницы
\usepackage{geometry}
\geometry{top=2cm, bottom=2cm, left=2.5cm, right=2.5cm}

% Гиперссылки
\usepackage{hyperref}

% Плавающие объекты
\usepackage{float}

% Дополнительные пакеты
\usepackage{venndiagram}

% Настройки заголовка
\title{Домашнее задание}
\author{Студент: \textbf{Ростислав Лохов}}
\date{\today}

\begin{document}

% Титульный лист
\begin{titlepage}
    \centering
    \vspace*{1cm}

    \Huge
    \textbf{Домашнее задание}

    \vspace{0.5cm}
    \LARGE
    По курсу: \textbf{Дискретная математика}

    \vspace{1.5cm}

    \textbf{Студент: Ростислав Лохов}

    \vfill

    \Large
    АНО ВО Центральный университет\\
    \vspace{0.3cm}
    \today

\end{titlepage}

% Содержание
\tableofcontents
\newpage

% Основной текст
\section{Примеры и контрпримеры, доказательство от противного}

ДИСКЛЕЙМЕР: Мне очень нужно попасть на красно-черный уровень, для этого нужно 8  баллов минимум набрать в течение 3х недель, не придирайтесь строго пожалуйста:) Я попал сюда потому что пропустил тест:(

\subsection{Задача 1}
\begin{enumerate}
    \item Предположим, что набора из трех бабочек где все цвета и все размеры не существует
    \item Построим таблицу 3 на 3, где строки - цвета, столбцы - размеры
    \item Тогда согласно условию задачи каждый столбец и каждая строка непусты
    \item Если бы не существовал набор из трёх бабочек, попарно различающихся по цвету и размеру, то невозможно было бы выбрать по одной бабочке из каждой строки и каждого столбца, что противоречит принципу Дирихле
    \item т.к предположение ведет к противоречию, то следовательно существуют три абсолютно различные бабочки.
\end{enumerate}

\subsection{Задача 2}
\begin{enumerate}
    \item Предположим, что студенты решили сговориться и решили выбрать следующим образом числа: $23, 23, 22, 22, 22, 22, 22, 22$
    \item Если лень считать - $2\cdot23 + 7\cdot22 = 200$
    \item Тогда рассмотрим идеальный вариант для Ильи: Илья выбирает максимальные числа для максимизации выигрыша.
    \item В таком случае он выберет 23, 23, 22, 22.
    \item Сумма таких чисел - 90, что меньше, чем 100
    \item Следовательно Илья не всегда может выбрать из них 4 числа так, что их сумма больше 100.
\end{enumerate}

\subsection{Задача 3}
\begin{enumerate}
    \item Предположим, что невозможно оставить коробки с одинаковым числом посылок (больше или равно 100 в сумме)
    \item Рассмотрим два случая, все коробки содержат > 100 посылок и < 100 посылок:
    \item Первый случай: больше или равно 100 посылок, если бы существовала коробка с m меньше чем 100 посылками, то можно было бы собрать несколько таких коробок с одинаковым количеством посылок m, чтобы их суммарное количество посылок было больше или равно 100. Например, для m=50, достаточно 2 коробок. Это противоречит предположению.
    \item Второй случай: кол-во коробок меньше 100. Тогда по принципу дирихле найдется хотя бы 2 коробки с одинаковым m.Eсли $m\cdot k \ge 100$ для некоторого k, то можно составить k коробок с m, что противоречит предположению.
    \item Из двух случаев следует, что коробок должно быть меньше или равно 20, т.к $\frac{2000}{100}=20$ но тогда все 20 коробок содержат 100 посылок, оставив их, получим требуемоую ситуацию, что противоречит исходному предположению. 
    \item Значит такая ситуация всегда достижима
\end{enumerate}

\subsection{Задача 4}
\begin{enumerate}
    \item Предположим, что такое правда существует, рассмотрим верхний левый угол квадрата
    \item Тогда согласно условию, существует закрашенная клетка под ним и справа от него. 
    \item Если (1,1) закрашена, её соседи (1,2) и (2,1) также должны быть закрашены (так как у угловой клетки только два соседа).
    \item Если (1,1) не закрашена, её соседи (1,2) и (2,1) обязаны быть закрашены, чтобы суммарно дать два закрашенных соседа для (1,1).
    \item Для клетки (1,2): она закрашена, поэтому у неё должно быть ровно два закрашенных соседа. Уже есть (2,1), значит, ещё один сосед (например, (1,3)) должен быть закрашен.
    \item Для клетки (2,1): аналогично, закрашиваем (3,1), чтобы у (2,1) было два закрашенных соседа.
    \item Продолжая эту логику, вдоль первой строки и первого столбца закрашиваются клетки через одну: (1,2), (1,4), (1,6), (1,8) и (2,1), (4,1), (6,1), (8,1).
    \item На пересечении закрашенных строк и столбцов возникает клетка (2,2). Она должна быть закрашена (так как соседствует с (1,2) и (2,1)), но у неё уже два закрашенных соседа. Если (2,2) закрашена, то у неё должно быть ровно два закрашенных соседа. Однако её соседи: (1,2), (2,1), (3,2), (2,3). Из них уже два закрашены ((1,2) и (2,1)), поэтому (3,2) и (2,3) не могут быть закрашены. Это нарушает условия для клеток (3,2) и (2,3), которые теперь не смогут иметь по два закрашенных соседа.
    \item Аналогичные рассуждения применяем к правому нижнему углу (9,9). Если закрасить клетки (9,8) и (8,9), то в клетке (8,8) возникнет противоречие: она должна быть закрашена (как пересечение строки и столбца), но её соседи (8,7), (7,8), (9,8), (8,9) уже включают два закрашенных, что делает невозможным выполнение условия.
    \item В квадрате 9×9 количество клеток в строке и столбце нечётное. При попытке закрасить клетки через одну (например, чётные позиции) в последней клетке строки/столбца (позиция 9) возникает конфликт: она должна быть закрашена, но её соседи уже нарушают условие.
    \item Требуемая раскраска не существует.
\end{enumerate}

\subsection{Задача 5}
\begin{enumerate}
    \item $\sum_{i=1}^{n}a_i < 0 \land \forall i(a_i+a_{i+1}+a_{i+2}) > 0$
    \item Суммируем все такие подпоследовательности, получим, что $a_1$ и $a_{20}$ встречаются по одному разу, $a_2, a_{19}$ встречаются 2 раза, все остальные - по 3 раза.
    \item $a_1 + a_2 + 3(a_3+...+a_{18}) + 2a_{19} + a_{20} > 0$
    \item Заметим, что S входит в полученное равенство, только с меньшими коэффициентами, в таком случае невозможно, чтобы сумма была отрицательной (возникает противоречие)
\end{enumerate}

\subsection{Задача 6}
Хорошо, минимизируем количество связей красный-красный, в таком случае будет последовательность вида красный-синий, в такой расстановке у каждого синего будет 2 соседа красных, а у каждого красного - два синих(тут должен был быть рисунок, но мне лень). В таком случае утверждение неверно. 


\subsection{Задача 7}
\begin{enumerate}
    \item Предположим обратное - не существует друзей которые подарили друг другу подарки
    \item Рассмотрим множество неупорядоченных пар друзей, его мощность - $\binom{2}{10} = 45$
    \item Рассмотрим количество подарков, каждый по 5, значит всего $5\cdot 10 = 50$
    \item т.к количество подарков больше, чем множество неупорядоченных пар друзей, то существуют такие друзья, которые подарили друг другу (по принципу Дирихле).
    \item Возникает противоречие, а значит существуют такие друзья, что дарят друг другу подарки.
\end{enumerate}

\subsection{Задача 8}
\begin{enumerate}
    \item Предплоложим обратное, все группы содержат не более 14 студентов, тогда минимальное количество групп - 5.
    \item Из каждой группы(14 человек) можно выбрать не более 2х студентов, чтобы избежать трех из одной группы. т.е 10 студентов.
    \item Таким образом существует набор из 10 студентов, где ни в одной группе нет трёх человек, что противоречит условию задачи.
\end{enumerate}

\end{document}