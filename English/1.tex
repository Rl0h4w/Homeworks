\documentclass[a4paper,12pt,english]{article}

% Encoding and language
\usepackage[utf8]{inputenc}
%\usepackage[russian]{babel} % Removed Russian language support

% Math packages
\usepackage{amsmath,amsfonts,amssymb}

% Graphics
\usepackage{graphicx}
\usepackage{tikz}
\usetikzlibrary{shapes.geometric, calc}
\usepackage{pgfplots}
\pgfplotsset{compat=1.18} % Added to suppress PGFPlots warning

% Page geometry
\usepackage{geometry}
\geometry{top=2cm, bottom=2cm, left=2.5cm, right=2.5cm}

% Hyperlinks
\usepackage{hyperref}

% Floating objects
\usepackage{float}

% Additional packages
\usepackage{venndiagram}

% Title settings
\title{Homework}
\author{Student: \textbf{Rostislav Lokhov}}
\date{\today}

\begin{document}

% Title page
\begin{titlepage}
    \centering
    \vspace*{1cm}

    \Huge
    \textbf{Homework}

    \vspace{0.5cm}
    \LARGE
    Course: \textbf{English}

    \vspace{1.5cm}

    \textbf{Student: Rostislav Lokhov}

    \vfill

    \Large
    ANO HE Central University\\
    \vspace{0.3cm}
    \today

\end{titlepage}


\subsection{Task 1}

\subsubsection*{Part a)}
\begin{enumerate}
    \item scroll
    \item lose
    \item going on a digital detox
    \item am addicted to
    \item compulsively
    \item affect
    \item spend
    \item go on a digital detox
    \item limiting my screen time
\end{enumerate}

\subsubsection*{Part b)}
\begin{enumerate}
    \item False
    \item False
    \item True
    \item True
\end{enumerate}

\subsubsection*{Part c)}
\paragraph*{AI generated paragraph:}
\texttt{Social media addiction can be a significant problem due to its pervasive nature and potential impact on mental and emotional well-being. For individuals prone to addiction, the constant validation-seeking and fear of missing out (FOMO) inherent in social platforms can trigger compulsive usage. This can lead to neglecting responsibilities, disrupted sleep patterns, and strained real-life relationships. For instance, a student might spend excessive hours scrolling feeds instead of studying, or an employee might constantly check notifications during work. To address this, individuals can implement strategies like setting time limits, muting notifications, or even taking digital detox periods. Seeking support groups or therapy can also be beneficial in overcoming this modern challenge.}

\paragraph*{Answers to the questions:}
\begin{enumerate}
    \item \textbf{Is the response fully relevant to the task? What parts may be irrelevant?} Yes, it is relevant and on topic. No irrelevant parts.

    \item \textbf{Analyse the word choice. Would you choose the same words? If not, what words would you remove / substitute with other words and why?} The words are good and quite formal, suitable for this task. I would use similar words.

    \item \textbf{Analyse the sentence structures. Do they look natural? What sentences would you simplify or shorten? Would you add more variety?} Sentences are natural and varied in length. They are easy to read. No need to change them.

    \item \textbf{If you were writing a similar text yourself, what aspects of the task would you approach differently? (e.g. choice of the example; would give a different answer, etc.)} I would write it similarly. The AI gave good problems and solutions. Maybe different examples, but the main points are good.

    \item \textbf{If you had to write such paragraph, how could you use AI ethically?} I can use AI to:
    \begin{itemize}
        \item Get ideas for problems and solutions.
        \item Find better words.
        \item Check if I missed anything.
        \item Get inspiration for structure.
        But the main writing should be my own.
    \end{itemize}
\end{enumerate}


\end{document}