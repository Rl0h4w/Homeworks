\documentclass[a4paper,12pt]{article}
\usepackage{amssymb}
\usepackage{amsfonts}

% Кодировка и язык
\usepackage[utf8]{inputenc}
\usepackage[T2A]{fontenc}
\usepackage[russian]{babel}


% Математические пакеты
\usepackage{amsmath,amsfonts,amssymb}
%Таблицы
\usepackage{array}
\usepackage{booktabs} % Для более красивых горизонтальных линий в таблицах
\usepackage{graphicx}
\usepackage{array}
\usepackage{booktabs}
% Геометрия страницы
\usepackage{geometry}
\geometry{top=2cm, bottom=2cm, left=2.5cm, right=2.5cm}
% Гиперссылки (лучше загружать последним)
\usepackage{hyperref}


% Настройки заголовка
\title{Домашнее задание}
\author{Студент: \textbf{Ростислав Лохов}}
\date{\today}

\begin{document}

% Титульный лист
\begin{titlepage}
    \centering
    \vspace*{1cm}

    \Huge
    \textbf{Домашнее задание}

    \vspace{0.5cm}
    \LARGE
    По курсу: \textbf{Экономика}

    \vspace{1.5cm}

    \textbf{Студент: Ростислав Лохов}

    \vfill

    \Large
    АНО ВО Центральный университет\\
    \vspace{0.3cm}
    \today

\end{titlepage}

% Содержание
\tableofcontents
\newpage

% Основной текст
\section{Сине-Красный уровень}


\subsection{Задача 1}
\begin{enumerate}
    \item Величина спроса увеличится на 0.8, выручка снизится на 1.2
    \item Величина предложения вырастет на 6, выручка увеличится на 9.18
    \item Спрос вырастет на 1.4, товар - нормальный
    \item Спрос на товар А снизится на 5, товары А и Б субституты
\end{enumerate}


\subsection{Задача 2}
\begin{center}
    \begin{tabular}{c c c c c c}
    P & Qd & $\Delta P$ & $\Delta Q$ & eD & TR \\ 
    \hline
    22,5 & 10  & -2.5 & 10 & -9 & 225 \\
    20,0 & 20  & -5.0 & 20 & -4 & 400 \\
    15,0 & 40  & -2.5 & 10 & -1.5 & 600 \\
    12,5 & 50  & -2.5 & 10 & -1 & 625 \\
    10,0 & 60  & -5.0 & 20 & -0.67 & 600 \\
    5,0  & 80  & -2.5 & 10 & -0.25 & 400 \\
    2,5  & 90  & -1.5 & 6 & -0.11 & 225 \\
    1,0  & 96  & -1   & 4 & -9 0.04  &96 \\
    0    & 100 &  0   & 0 & 0 & 0
    \end{tabular}
\end{center}

Выручка максимальна при 12.5 р за штуку 

Необходимо продать 50 тысяч единиц

\subsection{Задача 3}

\[
e = \frac{-300}{40}\cdot \frac{240}{1700} = -1.06
\]

Спрос эластичен т.к по модулю больше 1

\subsection{Задача 4}

\begin{enumerate}
    \item Эластичный спрос и эффект дохода( снижение цены - больше покупателей, скидки - увеличение обьема продаж)
    \item Дорогие часы - скидки снижают статус товара, покупатели премиум сегмента менее восприимчивы к изменению цены
\end{enumerate}

\subsection{Задача 5}
\begin{enumerate}
    \item Эффект отсрочки платежа - плачу потом, а не сейчас
    \item Эффект срочности - далее стоимость будет выше
    \item Эффект якоря - указание на большую скидку становится основной точкой сравнения
\end{enumerate}

\subsection{Задача 6}

\begin{enumerate}
    \item 2014: 35k rub
    \item 2015: 60k rub
    \item $dQ = -Q/2$
    \item $dQ/\overline{Q} = -2/3$
    \item $dP/\overline{P} = 25000/47500 = 0.526$
    \item $E = -1.27$
\end{enumerate}

\subsection{Задача 7}
\begin{enumerate}
    \item $e = \frac{dQ_D}{dP}\frac{P}{Q} = -x\cdot \frac{0.75}{7.5}=-0.8$
    \item $x = 8$
    \item $Q=a-bP = 13.5-8P$
\end{enumerate}

\begin{enumerate}
    \item $e_s = d\frac{P}{Q}$
    \item $d\frac{0.75}{7.5} = 1.6$
    \item $d=16$
    \item $Q_s = -4.5+16P$
\end{enumerate}

\subsection{Задача 8}
Поскольку яблочные девайсы становятся показателем статуса, их ца - зачастую люди, которые готовы заплатить любую сумму за девайс от такого производителя. (Престижный эффект) (+ когда покупаешь что-то премиальное, не хочется ощущать пластик, стекло ощущается дороже)
\section{Черный уровень}

\subsection{Задание 1}

\[
E_p = \frac{\partial Q_d}{\partial P} \frac{P}{Q_d} = \frac{-2P}{100000-2P+0.01\cdot 80000+0.2\cdot 2000} = -\frac{30000}{50600-30000} = \frac{300}{206} = 1.46
\]

\[
E_I = \frac{0.01\cdot 80000}{100000-2\cdot 30000+0.01\cdot 80000+0.2\cdot 2000} = 0.02
\]

\[
E_{P_v} = \frac{0.2\cdot 2000}{100000-2\cdot 30000+0.01\cdot 80000+0.2\cdot 2000} = 0.01
\]

Включил бы сезонность, очевидно, что кондей не нужен зимой, однако нужна еще температура, где зима жаркая - кондиционер нужен. 

\subsection{Задача 2}
\begin{enumerate}
    \item $Q(P) = 30-5.5P$ - суммарный рыночный спрос
    \item $R(P) = 30-5.5P^2$ - суммарное денежное предложение
    \item $\frac{dR}{dP} = 30-11P=0 \Rightarrow P=2.73$
    \item $Q(2.73)=15$
    \item 15 единиц за 2.73.
\end{enumerate}

\subsection{Задача 3}
\begin{enumerate}
    \item Тяжелая финансовая ситуация в стране
    \item Ограниченный рынок - те, кто летали, летали по любой цене(деловые люди)
    \item В общем на спрос влияли неценовые факторы, а потому скидка в 50\% не увеличила пассажиропоток
\end{enumerate}

\subsection{Задача 4}
\begin{enumerate}
    \item Повысить цену - минусы: вероятность уехать в лес не по своей воле возрастает(этичесикй аспект), вмешательство государства, и т.п. Плюсы - Прибыль максимальная
    \item Постепенно повышать цену - минусы: сложность реализации. Плюсы - стабильность
    \item Оставить цену - всем выгодно, нам - репутация, им - хорошие цены.
\end{enumerate}
\end{document}