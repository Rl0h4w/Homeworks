\documentclass[a4paper,12pt]{article}
\usepackage{amssymb}
\usepackage{amsfonts}

% Кодировка и язык
\usepackage[utf8]{inputenc}
\usepackage[T2A]{fontenc}
\usepackage[russian]{babel}


% Математические пакеты
\usepackage{amsmath,amsfonts,amssymb}
%Таблицы
\usepackage{array}
\usepackage{booktabs} % Для более красивых горизонтальных линий в таблицах
\usepackage{graphicx}
\usepackage{array}
\usepackage{booktabs}
% Геометрия страницы
\usepackage{geometry}
\geometry{top=2cm, bottom=2cm, left=2.5cm, right=2.5cm}
% Гиперссылки (лучше загружать последним)
\usepackage{hyperref}


% Настройки заголовка
\title{Домашнее задание}
\author{Студент: \textbf{Ростислав Лохов}}
\date{\today}

\begin{document}

% Титульный лист
\begin{titlepage}
    \centering
    \vspace*{1cm}

    \Huge
    \textbf{Домашнее задание}

    \vspace{0.5cm}
    \LARGE
    По курсу: \textbf{Экономика}

    \vspace{1.5cm}

    \textbf{Студент: Ростислав Лохов}

    \vfill

    \Large
    АНО ВО Центральный университет\\
    \vspace{0.3cm}
    \today

\end{titlepage}

% Содержание
\tableofcontents
\newpage

% Основной текст
\section{Сине-Красный уровень}

\subsection{Задача 1}
\begin{enumerate}
    \item Допустим продавцы роз договорились не снижать цены на рынке, чтобы сохранить высокую прибыль.
    \item Однако один из них снижает цену, чтобы увеличить долю рынка. Т.е если оба сотрудничают - оба получают хорошую прибыль. Если один предаёт - он получает сверхприбыль. Если оба предают, то прибыль снижается.
    \item Рациональный выбор - предать, что приводит к равновесию Нэша.
\end{enumerate}

\subsection{Задача 2}
\begin{enumerate}
    \item Используя формулу Лернера: $\frac{P-MC}{P}=\frac{1}{|E|}$
    \item Для первой группы: $\frac{5000-3000}{5000}=0.4 \Rightarrow |E| = 2.5$ - эластичность
    \item Для второй $\frac{4000-3000}{4000} = 0.25 \Rightarrow |E|=4$ - эластичность
\end{enumerate}

\subsection{Задача 3}
\begin{enumerate}
    \item Авиакомпании использую дискриминацию третьего типа(сегментация) т.к деловые пассажиры готовы платить больше и не задерживаются на выходных, в отличие от туристов. Скидка для тех, кто летит в воскресенье позволяет отделить бюджетных клиентов от премиальных
\end{enumerate}

\subsection{Задача 4}
\begin{enumerate}
    \item $MC = 20, P=50, Q=20, MR_1 = 40, Q=20, MR_2=10, Q=20$
    \item Налог - разница между $MR_1$ и $MC$ = 20
    \item Субсидия - разница между MC и $MR_2$ = 10
    \item Налог в 20 увеличит MC до 40, что совпадет с MR выше излома. Фирма перестанет считать выгодным держать цену 50 и повысит её.
    \item Субсидия в 10 снизит MC до 10, что совпадет с MR ниже излома. Фирма снизит цену, чтобы увеличить объём продаж.
\end{enumerate}

\subsection{Задача 5}
\begin{enumerate}
    \item \includegraphics[scale=0.45]{graphs/8.1.png}
\end{enumerate}

\subsection{Задача 6}
\begin{enumerate}
    \item Аукциона на eBay - индивидуальные цены
    \item Тарифы премиум и эконом - самоотбор
    \item Скидки студентам - имба(сегментация)
    \item Маркетологи создают усилия, где клиенты сами выбирают тарифы в зависимости от готовности платить. Например премиум пасс на американские горки - пропуск всех очередей.
\end{enumerate}

\subsection{Задача 7}
\begin{enumerate}
    \item Для начала скажем, что это не потому что Вася, Петя, Коля и так 99 процентов населения решили продать акции, потому что их кумир решил сделать какое то высказывание/действие в сторону компании.
    \item Даже скорее так, 1 процент от общего числа, который владеет, пусть будет 80 процентами всех ресурсов на земле.
    \item Каждый человек, который входит в такое множество скорее всего владеет венчурными/инвестиционным и т.п фондом, который хорошо автоматизирован под такие задачи и читает все возможные новости в секунду.
    \item Данная новость могла восприняться как изменение потребительских трендов, полит влияние на бизнес, потеря соц влияния платформы.
    \item И начинается перебалансировка портфеля
\end{enumerate}

\section{Черный уровень}

\subsection{Задача 1}
\begin{enumerate}
    \item $Q = 110 - P, AC = MC = 10$ $q_1$ - обьем производства DL, $q_2$ - обьем производства Angels
    \item Общий обьем $q_1+q_2 \rtimes Q = 110-(q_1+q_2)$
    \item Последователь, Angels, Inc., максимизирует свою прибыль, принимая объем производства лидера, $q_1$ как заданный. Прибыль последователя составляет:
    \item $\pi_2 = P \cdot q_2 - MC \cdot q_2 = (110 - q_1 - q_2)q_2 - 10q_2 = 110q_2 - q_1q_2 - q_2^2 - 10q_2 = 100q_2 - q_1q_2 - q_2^2$
    \item Для максимизации прибыли возьмем производную $\pi_2$ по $q_2$ и приравняем к нулю:
    \item $\frac{d\pi_2}{dq_2} = 100 - q_1 - 2q_2 = 0$
    \item $q_2 = 50-0.5q_1$
    \item Далее определим обьем производства лидера.
    \item $\pi_1 = P \cdot q_1 - MC \cdot q_1 = (110 - q_1 - q_2)q_1 - 10q_1$
    \item $\pi_1 = \left(110 - q_1 - \left(50 - \frac{1}{2}q_1\right)\right)q_1 - 10q_1 = \left(110 - q_1 - 50 + \frac{1}{2}q_1\right)q_1 - 10q_1 = \left(60 - \frac{1}{2}q_1\right)q_1 - 10q_1 = 60q_1 - \frac{1}{2}q_1^2 - 10q_1 = 50q_1 - \frac{1}{2}q_1^2$
    \item $\frac{d\pi_1}{dq_1} = 50 - q_1 = 0$
    \item $q_1 = 50$
    \item $q_2 = 50 - \frac{1}{2}(50) = 50 - 25 = 25$
    \item $P = 110 - (q_1 + q_2) = 110 - (50 + 25) = 110 - 75 = 35$
    \item $\pi_1 = (P - MC)q_1 = (35 - 10) \cdot 50 = 25 \cdot 50 = 1250$
    \item $\pi_2 = (P - MC)q_2 = (35 - 10) \cdot 25 = 25 \cdot 25 = 625$
\end{enumerate}

\subsection{Задача 2}
\begin{enumerate}
    \item $VC_i = Q^2 + 3Q$
    \item $MC_i = \frac{d(VC_i)}{dQ} = 2Q + 3$
    \item $Q_i = \frac{P - 3}{2}$
    \item $Q_s = n \cdot Q_i = n \cdot \frac{P - 3}{2}$
    \item $Q_d = Q - Q_s = (45 - 5P) - n \cdot \frac{P - 3}{2}$
    \item $Q_d = 45 - 5P - \frac{nP}{2} + \frac{3n}{2}$
    \item $Q_d = \left(45 + \frac{3n}{2}\right) - \left(5 + \frac{n}{2}\right)P$
    \item $\left(5 + \frac{n}{2}\right)P  = \left(45 + \frac{3n}{2}\right) - Q_d$
    \item $P = \frac{\left(45 + \frac{3n}{2}\right) - Q_d}{\left(5 + \frac{n}{2}\right)}$
    \item $P = \frac{90 + 3n - 2Q_d}{10 + n}$
    \item $P = \frac{90 + 3n - 2Q}{10 + n}$
    \item $TR = PQ = \frac{(90+3n-2Q)Q}{10+n}$
    \item $MR = \frac{dTR}{dQ}=\frac{90+3n-4Q}{10+n}$
    \item $MR = MC \Rightarrow MC = 2Q+3$
\end{enumerate}
\begin{align*}
    \frac{(90 + 3n) - 4Q}{10 + n} &= 2Q + 3 \\
    (90 + 3n) - 4Q &= (10 + n)(2Q + 3) \\
    90 + 3n - 4Q &= 20Q + 30 + 2nQ + 3n \\
    60 &= 24Q + 2nQ \\
    60 &= (24 + 2n)Q \\
    Q &= \frac{60}{24 + 2n} \\
    Q &= \frac{30}{12 + n}
\end{align*}

\begin{align*}
    Q &= \frac{30}{12 + 18} = \frac{30}{30} = 1 \\
    P &= \frac{90 + 3n - 2Q}{10 + n} \\
    P &= \frac{90 + 3 \cdot 18 - 2 \cdot 1}{10 + 18} \\
    P &= \frac{90 + 54 - 2}{28} \\
    P &= \frac{142}{28} \\
    P &= \frac{71}{14} \approx 5.07
\end{align*}

\end{document}