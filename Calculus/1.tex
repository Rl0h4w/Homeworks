\documentclass[a4paper,12pt]{article}

% Кодировка и язык
\usepackage[utf8]{inputenc}
\usepackage[russian]{babel}

% Математические пакеты
\usepackage{amsmath,amsfonts,amssymb}

% Графика
\usepackage{graphicx}
\usepackage{tikz}
\usetikzlibrary{shapes.geometric, calc}
\usepackage{pgfplots}
\pgfplotsset{compat=1.18} % Добавлено для устранения предупреждения

% Геометрия страницы
\usepackage{geometry}
\geometry{top=2cm, bottom=2cm, left=2.5cm, right=2.5cm}

% Гиперссылки
\usepackage{hyperref}

% Плавающие объекты
\usepackage{float}

% Дополнительные пакеты
\usepackage{venndiagram}

% Настройки заголовка
\title{Домашнее задание}
\author{Студент: \textbf{Ростислав Лохов}}
\date{\today}

\begin{document}

% Титульный лист
\begin{titlepage}
    \centering
    \vspace*{1cm}

    \Huge
    \textbf{Домашнее задание}

    \vspace{0.5cm}
    \LARGE
    По курсу: \textbf{Математический Анализ}

    \vspace{1.5cm}

    \textbf{Студент: Ростислав Лохов}

    \vfill

    \Large
    АНО ВО Центральный университет\\
    \vspace{0.3cm}
    \today

\end{titlepage}

% Содержание
\tableofcontents
\newpage

% Основной текст
\section{Открытые и замкнутые множества в $\mathbb{R}^n$}

\subsection{Задача 1}

\begin{enumerate}
    \item Граничные точки — такая точка, что 
    \[
    (\forall \varepsilon > 0) 
    \begin{cases}
        U_{\varepsilon}(x) \cap A \ne \emptyset, \\ 
        U_{\varepsilon}(x) \cap (X\setminus A) \ne \emptyset.
    \end{cases}
    \]
    
    Тоесть простыми словами, окрестность точки или сама точка имеет пересечение со всем множеством A и окрестность имеет пересечение с надмножеством А.
    
    
    \item Предельная точка — $(\forall \varepsilon > 0) U_{\varepsilon} \cap (A \setminus \{x\} \ne \emptyset)$
    
    \item Точки прикосновения - $(\forall \varepsilon > 0) U_{\varepsilon} \cap A \ne \emptyset$
    
    \item Внутренняя точка - $(\exists \varepsilon > 0) U_{\varepsilon}(x) \subset A$
    
    \item Граничные точки E: $\{0, 1, 2, 3, 4, 5, [8;11]\}$
    \item Предельные точки E: $\{E \setminus (\{2\} \cup \{5\}) \}$
    \item Точки прикосновения E:  E
    \item Внутренние точки - $(0;1) \cup (3;4)$
\end{enumerate}

\subsection{Задача 2}

Предельная точка - $(\forall \varepsilon > 0) (U_{\varepsilon}\cap A\setminus \{x\} \ne \emptyset)$

Тогда согласно условию задачи $\forall x \in E \exists c: (\forall \varepsilon > 0) (U_{\varepsilon}\cap A\setminus \{c\} \ne \emptyset)$ что противоречит само себе т.к множество изолированно, т.е $c: U_{\varepsilon} \cap A = \emptyset$


\subsection{Задача 3}
Множество называется открытым, если все его точки внутренние т.е для всех точек существует окрестность в которой они полностью лежат в множестве.

Множество называется замкнутым, если оно содержит все свои точки прикосновения т.е содержит все такие точки что пересечение их окрестностей с рассматриваемым множеством ненулевое.

Да, $[0;1)$

\subsection{Задача 4}

Множество называется замкнутым, если оно содержит все свои точки прикосновения т.е содержит все такие точки, что пересечение их окрестности с рассматриваемым множеством ненулевое.

Граница множества - такая точка, что 
    \[
    (\forall \varepsilon > 0) 
    \begin{cases}
        U_{\varepsilon}(x) \cap A \ne \emptyset, \\ 
        U_{\varepsilon}(x) \cap (X\setminus A) \ne \emptyset.
    \end{cases}
    \]

\begin{enumerate}
    \item Предположим, что верна формула $\delta E = \overline{E} \cap \overline{X \setminus E}$
    \item Пусть $x \in \delta E$ тогда по определению границы множества выполняется. Также $x \in \overline{E} \land x \in \overline{X \setminus E}$
    \item Обратно пусть $x \in \overline{E} \cap \overline{X \setminus E}$ Тогда каждая окрестность U точки x пересекаются с E и пересекается с $X \setminus E$. Также удовлетворяет
    \item Таким образом верно.
\end{enumerate}

\subsection{Задача 5}
Изолированная точка - такая точка, окрестности которой не пересекаются с множеством, однако точка принадлежит множеству.

Тогда в окрестности каждой такой точки можно выбрать рациональное число. Таким образом можно установить биекцию, а т.к множество рациональных чисел счётно, то множетсво изолированных точек счётно.
\subsection{Задача 6}
\begin{enumerate}
    \item $x\sin(y)\ge 0$, $[0;+\infty]$
    \item Замкнуто, не является областью, т.к область должна быть открытой и связанной.
\end{enumerate}

\section{Предел по множеству}

\subsection{Задача 7}

\[
x = t\cos(\alpha) \land y = t\sin(\alpha) \land t = \sqrt{x^2+y^2}
\]

\[
\lim_{t \to +\infty} (t\cos(\alpha))^4e^{tsin(\alpha)-(t\cos(\alpha))^2}
\]

\[
\lim_{t \to +\infty} \frac{t^4(1-\sin^2(\alpha))^2}{e^{(t\cos(\alpha))^2-tsin(\alpha)}} = \lim_{t \to \infty} \frac{t^4\cos^2(\alpha)}{e^{t(t\cos^2(\alpha)-\sin(\alpha))}}
\]

применим правило Лопиталя дважды, получим
\[
\lim_{t \to \infty} \frac{12t^2}{e^{t(t\cos(\alpha)^2-\sin(\alpha))}(4t^2\cos^4(\alpha)+\cos(\alpha)^2(2-4\sin(\alpha))+\sin^2(\alpha))} = 0
\]

таким образом предел по направлению существует, однако общего предела нет(предела по совокупности) Контрпример:

\[
y=x^2 \Rightarrow \lim_{x \to \infty} x^4 = \infty
\]

а т.к пределы по направлению не совпадают, то предела не существует
\subsection{Задача 8}

\[
x = r\cos(\theta) \land y = r\sin(\theta) \land r = \sqrt{x^2+y^2}
\]

\[
\lim_{(x;y) \to (+\infty;+\infty)} \frac{x^2+3y^2}{e^{x+y}} = \lim_{r \to +\infty} \frac{(r\cos(\theta))^2+3(r\sin(\theta))^2}{e^{r(\cos(\theta)+\sin(\theta))}}
\]

т.к в знаменателе сумма в показателе может быть как отрицательна, так и положительна, то предел может быть как бесконечность, так и 0 соответственно.

\subsection{Задача 9}

Перейдем также к полярным координатам и покажем, что зависит от $\theta$:
\[
    x = r\cos(\theta) \land y = r\sin(\theta) \land r = \sqrt{x^2+y^2}
\]

\begin{enumerate}
    \item[a)] 
        \[
            \lim_{r \to 0}\frac{r\cos(\theta)(r\sin(\theta))^3}{(r\cos(\theta))^4+3(r\cos(\theta))^2(r\sin(\theta))^2+2(r\sin(\theta))^4} = \frac{-2\cos(\theta)\sin^3(\theta)}{-3+\cos(2\theta)}
        \]
        
        Таким образом видно, что предел зависит от $\theta$, а значит предела не существует.
    
    \item[b)]
        Рассмотрим различные случаи стремления к 0. Рассмотрим два случая с параметром, $y=kx$ и $y=mx^5$:
        
        \[
            \lim_{(x;kx)\to(0;0)} \frac{x^6k}{x^{10}+2k^2x^2} = 0
        \]
        
        \[
            \lim_{(x;kx^5)\to(0;0)} \frac{x^11k}{x^{10}+2k^2x^10} = \frac{m}{1+2m^2}
        \]

        Т.к во втором примере предел зависит от константы, и пределы различны в зависимости от скорости с которой мы приближаемся к 0, в таком случае предела нет.

        


\end{enumerate}

\section{Поиск предела с помощью перехода в полярные координаты}

\subsection{Задача 10}

\[
x = r\cos(\theta) \land y = r\sin(\theta) \land r = \sqrt{x^2+y^2}
\]

\[
\lim_{r \to 0} \frac{r\sin(\theta)\sin(r\cos(\theta))-r\cos(\theta)\sin(r\sin(\theta))}{((r\cos(\theta))^2+(r\sin(\theta))^2)^2}
\]

\[
\lim_{r \to 0} \frac{\sin(\theta) \sin(r \cos(\theta)) - \cos(\theta) \sin(r \sin(\theta))}{r^3}
\]

\[
\lim_{r \to 0}\frac{\sin(\theta) \left( r \cos(\theta) - \frac{r^3 \cos^3(\theta)}{6} \right) - \cos(\theta) \left( r \sin(\theta) - \frac{r^3 \sin^3(\theta)}{6} \right) + o(r^3)}{r^3} = -\frac{1}{6} \sin\theta \cos\theta \cos(2\theta)
\]

Таким образом предел зависит от $\theta$, а значит предела нет

\section{Повторный предел}

\subsection{Задача 11}
\[
\lim_{x \to 0} f(x;y) = -1
\]

\[
\lim_{y \to 0} f(x;y) = 1
\]

\subsection{Задача 12}
\[
\begin{cases}
    f(x;y) = ysin(\frac{1}{x}), x \ne 0 \\
    f(x;y) = 0, x = 0
\end{cases}
\]
имеет предел по совокупности, однако не существует обоих повторных пределов

\subsection{Задача 13} 
По определению предел по совокупности переменных - все переменные одновременно стремятся к какой то точке и, согласно условию задачи, предел существует.
Определение повторного предела - пределы по одной переменной, а значит, если существует предел по совокупности, то существует и предел по каждой переменной, т.е повторный предел

\section{Примеры исследования пределов}

\subsection{Задача 14}

\begin{enumerate}
    \item[a)]   \[
                \lim_{r \to 0} \frac{r^4\cos(\varphi)\sin(\varphi)^3}{\sqrt{r^6(\cos^6(\varphi)+\sin^6(\varphi))}}
                \]
                сделаеем оценку, не зависящую от $\varphi$
                \[
                |f(x_0+r\cos(\varphi); y_0 + r\sin(\varphi)| \le r
                \]
                Таким образом, оценка существует, а значит существует и предел по совокупности.
    \item[b)]   \[
                \lim_{r \to 0} \frac{\frac{r^2\cos(\varphi)^2}{2}}{\sqrt{r^2(\cos^2(\varphi)+\sin^2(\varphi))}}=\frac{r\cos(\varphi)^2}{\sqrt{\cos^2(\varphi)+\sin^2(\varphi)}} = 0
                \]
                \[
                |f(x_0+r\cos(\varphi); y_0 + r\sin(\varphi)| \le |r|
                \]
                Таким образом, оценка существует, а значит существует и предел по совокупности.
\end{enumerate}


\section{Непрерывность по множеству, совокупности и по переменной}

\subsection{Задача 15}
Определение непрерывной функции - функция называется непрерывной, если $\lim_{(x,y) \to (x_0;y_0)}f(x, y)=f(x_0;y_0)$

Проверим в точке разрыва $0=\lim_{(x,y) \to (0;0)} 0\sin(\frac{1}{0})$ т.к произведение ограниченной на бесконечно малую то получим 0.  

т.е непрерывна в 0. В остальных точках она непрерывна т.к 1/x разрывна в 0, sin(x) неразрывен.

\subsection{Задача 16}
Аналогично проверим. Только теперь у нас будет что-то вроде линии разрыва. $x=-y$

\[
\lim_{x \to -y} \frac{(x+y)(x^2-xy+y^2)}{x+y} = 3y^2 
\]

\[
3y^2=3, y=1 \Rightarrow x =-1 \lor y=-1 \Rightarrow x=1
\]

Т.е в случае если сумма стремится к 0, точками должны быть $(1;-1) \lor (-1;1)$

В любой другой точке функция разрывна если сумма равна 0

Доопределение: 

Можем просто сделать более сильное условие чтобы она была всюду неразрывна, тогда необходимость в остальных случаях отпадет: 

\[
f(x;y) = x^2-xy+y^2
\]

\section{Свойства непрерывных функций}

\subsection{Задача 17}
Абсолютно также:

\[
\lim_{x^2 \to 1-y^2} (1-y^2+y^2-1)\sin(\frac{1}{1-x^2-y^2})
\]

окей, в таком случае произведение ограниченной на бесконечно малую даёт 0, а т.к значение в этой же точке совпадает с пределом значит разрыва нет.

Хорошо, тогда рассмотрим когда обе координаты стремятся к бесконечности, в таком случае аргумент синуса стремится к 0, а синус 0 это 0, значит функция равна 0 на бесконечности по обоим аргументам одновременно, также и если мы будем рассматривать предел по каждому аргументу одновременно. 

Точек разрыва нет.

\subsection{Задача 18}
Здесь будет проще перейти к полярным координатам т.к можем уйти в комплексное поле.
\[
x = r\cos(\theta) \land y = r\sin(\theta) \land r = \sqrt{x^2+y^2} 
\]

\[
\lim_{r \to 0} r^{r^2\cos(\varphi)\sin(\varphi)}=1
\]
Произведение непрерывной на огриниченную в показателе даёт 0, причем показатель идет к 0 быстрее, чем числитель, а значит предел к этой точке равен значению в этой точке. во всех остальных точках функция непрерывна.
\end{document}