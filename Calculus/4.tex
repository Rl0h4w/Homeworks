\documentclass[a4paper,12pt]{article}

% Кодировка и язык
\usepackage[utf8]{inputenc}
\usepackage[russian]{babel}

% Математические пакеты
\usepackage{amsmath,amsfonts,amssymb}

% Графика
\usepackage{graphicx}
\usepackage{tikz}
\usetikzlibrary{shapes.geometric, calc}
\usepackage{pgfplots}
\pgfplotsset{compat=1.18} % Добавлено для устранения предупреждения

% Геометрия страницы
\usepackage{geometry}
\geometry{top=2cm, bottom=2cm, left=2.5cm, right=2.5cm}

% Гиперссылки
\usepackage{hyperref}

% Плавающие объекты
\usepackage{float}

% Дополнительные пакеты
\usepackage{venndiagram}

% Настройки заголовка
\title{Домашнее задание}
\author{Студент: \textbf{Ростислав Лохов}}
\date{\today}

\begin{document}

% Титульный лист
\begin{titlepage}
	\centering
	\vspace*{1cm}

	\Huge
	\textbf{Домашнее задание}

	\vspace{0.5cm}
	\LARGE
	По курсу: \textbf{Математический Анализ}

	\vspace{1.5cm}

	\textbf{Студент: Ростислав Лохов}

	\vfill

	\Large
	АНО ВО Центральный университет\\
	\vspace{0.3cm}
	\today

\end{titlepage}

% Содержание
\tableofcontents
\newpage

% Основной текст
\section{Примеры задач(безусловной) оптимизации}

\subsection{Задача 1}

\[
	u(x; y) = y^3 + 3yx^2-39y-36x+26
\]

\[
	du = 3y^2 dy + 3 x^2 dy + 6 xy dx - 39dy -36dx
\]

\[
    \begin{cases}
        3y^2+3x^2-39 = 0 \\
        6xy-36 = 0
    \end{cases}
\]

\[
(2, 3) \land (3, 2) \land (-3, -2) \land (-2, -3)
\]

\[
	d^2u = 6ydy^2 + 6xdxdy + 6(xdy+ydx)dx 
\]

\[
d^2u = 6ydy^2+12x dxdy+ 6y dx^2
\]

\[
\begin{pmatrix}
	6y & 6x \\
	6x & 6y
\end{pmatrix}
\]

\begin{enumerate}
	\item (2, 3): $\delta_1 = 12$, $\delta_2 = 180$ - положительно определена - минимум
	\item (3, 2): $\delta_1 = 18$, $\delta_2 = -180$ - неопределена
	\item (-2, -3): $\delta_1 = -12$, $\delta_2 = 180$ - отрицательно определена - максимум
	\item (-3, -2): $\delta_1 = -18$, $\delta_2 = -180$ - неопределена
\end{enumerate}

\subsection{Задача 2}

\[
u(x; y; z) = \frac{1}{z} + \frac{z}{y} + \frac{y}{x} + x + 1
\]

\[
du = -\frac{1}{z^2}dz + \frac{ydz-zdy}{y^2} + \frac{xdy - ydx}{x^2} + dx 
\]

\[
du = dx + \frac{dy}{x} - \frac{y}{x^2} dx - \frac{dz}{z^2} + \frac{dz}{y} - \frac{z}{y^2} dy
\]

\[
du = \left(1 - \frac{y}{x^2} \right) dx + \left(\frac{1}{x} - \frac{z}{y^2} \right) dy + \left(\frac{1}{y} - \frac{1}{z^2} \right) dz
\]

\[
\begin{cases}
	1 - \frac{y}{x^2} = 0 \\
	\frac{1}{x} - \frac{z}{y^2} = 0 \\
	\frac{1}{y} - \frac{1}{z^2}  = 0
\end{cases}
\]

\[
(-1, 1, -1) \land (1, 1, 1)
\]

\[
\begin{pmatrix}
	\frac{2y}{x^3} & \frac{-1}{x^2} & 0 \\
	-\frac{1}{x^2} & \frac{2z}{y^3} & \frac{-1}{y^2} \\
	0 & \frac{-1}{y^2} & \frac{2}{z^3}
\end{pmatrix}
\]

\[
H(1, 1, 1) = 
\begin{pmatrix}
	2& -1& 0 \\
	-1& 2& -1 \\
	0 &-1& 2 
\end{pmatrix}
\]

\[
H(-1, 1, -1) = 
\begin{pmatrix}
	-2& -1& 0 \\
	-1& -2& -1 \\
	0 &-1& -2 
\end{pmatrix}
\]

\begin{enumerate}
	\item H(1, 1, 1)  - положительно определенная, все три алгебраических дополнения больше 0 - локальный минимум
	\item H(-1, 1, -1) - отрицательно определенная, алгебраические дополнения чередуются - локальный максимум
\end{enumerate}



\subsection{Задача 3}
\[
u(x;y): x^3-y^2+u^2-3x+4y+u-8=0
\]

\[
3x^2dx-2ydy+2udu-3dx+4dy+du
\]

т.к в стационарных точках du = 0, то

\[
3x^2dx-2ydy-3dx+4dy = 0
\]

\[
(3x^2-3)dx + (4-2y)dy = 0
\]

\begin{enumerate}
	\item $1, 2, 2$
	\item $1, 2, -3$
	\item $-1, 2, 1$
	\item $-1, 2, -2$
\end{enumerate}

\[
6xdx^2-2dy^2+2du^2
\]

\[
H = 
\begin{pmatrix}
	6x & 0 & 0 \\
	0 & -2 & 0 \\
	0 & 0 & 2
\end{pmatrix}
\]

По критерию сильвестра найдем определенность матрицы: 

\begin{enumerate}
	\item $1, 2, 2$: $\delta_1 = 6, \delta_2 = -12, \delta_3 = -24$ - неопределена
	\item $1, 2, -3$: $\delta_1 = 6, \delta_2 = -12, \delta_3 = -24$ - неопределена
	\item $-1, 2, 1$: $\delta_1 = -6, \delta_2 = +12, \delta_3 = -24$ - отрицательно определенная, локальный максимум
	\item $-1, 2, -2$: $\delta_1 = -6, \delta_2 = +12, \delta_3 = -24$ - отрицательно определенная, локальный максимум
\end{enumerate}

\subsection{Задача 4}
\[
(x-y)^2+4x-4y+5
\]

\[
((x-y)^2+4(x-y)+4)+1 = u
\]

\[
((x-y)+2)^2+1 = u
\]

функция имеет только глобальный минимум u=1

\subsection{Задача 5}

\[
u = x^4+y^4-2x^2
\]

\[
du = 4x^3dx +4y^3dy-4xdx
\]

\[
du = (4x^3-4x)dx+4y^3dy
\]

\[
\begin{cases}
	4x(x^2-1) = 0 \\
	y^3=0
\end{cases}
\]

\[
x, y = (-1, 0), (0, 0), (1, 0)
\]

\[
d^2u = 12x^2d^2x+12y^2d^2y-4d^2x
\]

\[
d^2u = (12x^2-4)d^2x + 12y^2d^2y
\]

\[
H = \begin{pmatrix}
	12x^2-4 & 0 \\
	0 & 12y
\end{pmatrix}
\]

рассмотрим бесконечно малые приращения в точке (0, 0) отдельно по каждому аргументу: 

\[
u(\delta, 0) = \delta^4
\]

\[
u(0, \delta) = \delta^4-2\delta^2  = \delta^2(\delta^2-2)
\]

при $\delta$ от 0 до $\sqrt{2}$ знаки будут отличаться, значит не экстремум.

Далее попробуем найти решение аналитически.  Разделим график на две обьемные фигуры и спроецируем их на плостость uy:

В таком случае графиком будет гипербола и минимумом функции будет являться u = -1. В точке u=-1 x принимает значения $\pm 1$, а y - 0. Таким образом (-1, 0), (1, 0) - экстремумы

\subsection{Задача 6}

\[
||Ax-b||^2 +\lambda^2||x||^2= (Ax-b)^T(Ax-b) +\lambda x^Tx= x^TA^TAx-b^TAx-x^TA^Tb+b^Tb +\lambda^2 x^Tx= 
\]

\[
L(x) = x^T(A^TA+\lambda^2I)x-2x^TA^Tb+b^Tb
\]

\[
\nabla L(x) = 2(A^TA+\lambda^2I)x-2A^Tb = 0
\]

Если матрица A обратима:
\[
x = (A^TA+\lambda^2I)^{-1}A^Tb
\]

Т.к х существует если матрица обратима и функция имеет единственный глобальный минимум (функция регрессии с регуляризацией) то матрица положительно определена, следовательно минимум - х

\subsection{Задача 7}	

\[
\begin{cases}
	Q(k, l) = 50k^2l^{0.5} \\
	k=80-l
\end{cases}
\]

\[
	\begin{cases}
		Q(k, l) = 50(80-l)^2l^{0.5}\\
		k=80-l
	\end{cases}
\]

\[
dl = \frac{125(x^2-96x+1280)}{\sqrt{x}} = 0
\]

\[
l = 16 \Rightarrow k = 64
\]

\subsection{Задача 8}

\[
\begin{cases}
	x = \pm\sqrt{9+y^2} \\
	u(y) = 6 \mp 5\sqrt{9+y^2} - 4y
\end{cases}
\]

\[
	\begin{cases}
		x = \pm\sqrt{9+y^2} \\
		du = -4 \mp \frac{5y}{\sqrt{9+y^2}} = 0
	\end{cases}
\]

\[
\begin{cases}
	x = \pm \sqrt{9+y^2} \\
	y = \pm 4
\end{cases}
\]

\[
x = \pm 5 \land y = \pm 4
\]

\[
x, y = (5, 4) \lor x, y = (4, 5)
\]

\subsection{Задача 11}
\[
\begin{cases}
	u=(x-y)(x+y) \\
	x^2-2x+1+y^2\le 1
\end{cases}
\]
исследуем на экстремумы:

\[
du = 2xdx+2ydy
\]

\[
d^2u = 2d^2x-2d^2y
\]

т.е в точке (0, 0) не существует экстремума - седловая точка


Воспользуемся методом Лагранжа для исследования границ

\[
L(x, y, \lambda) = x^2-y^2-\lambda((x-1)^2y^2-1)
\]

при y=0 x=2 или x=0 u=4 или u = 0

при $\lambda = -1$ x = 0.5 $y = \pm 0.5\sqrt{3}$ $u=-0.5$

Внутренняя точка 0, граничные точки - 4, 0, -0.5

Максимум 4 (2, 0), минимум -0.5 (0.5, $\pm 0.5\sqrt{3}$)


\end{document}