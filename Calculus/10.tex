\documentclass[a4paper,12pt]{article}

% Кодировка и язык
\usepackage[utf8]{inputenc}
\usepackage[russian]{babel}

% Математические пакеты
\usepackage{amsmath,amsfonts,amssymb}

% Графика
\usepackage{graphicx}
\usepackage{tikz}
\usetikzlibrary{shapes.geometric, calc}
\usepackage{pgfplots}
\pgfplotsset{compat=1.18} % Добавлено для устранения предупреждения

% Геометрия страницы
\usepackage{geometry}
\geometry{top=2cm, bottom=2cm, left=2.5cm, right=2.5cm}

% Гиперссылки
\usepackage{hyperref}

% Плавающие объекты
\usepackage{float}

% Дополнительные пакеты
\usepackage{venndiagram}

% Настройки заголовка
\title{Домашнее задание}
\author{Студент: \textbf{Ростислав Лохов}}
\date{\today}

\begin{document}

% Титульный лист
\begin{titlepage}
	\centering
	\vspace*{1cm}

	\Huge
	\textbf{Домашнее задание}

	\vspace{0.5cm}
	\LARGE
	По курсу: \textbf{Математический Анализ}

	\vspace{1.5cm}

	\textbf{Студент: Ростислав Лохов}

	\vfill

	\Large
	АНО ВО Центральный университет\\
	\vspace{0.3cm}
	\today

\end{titlepage}

% Содержание
\tableofcontents
\newpage

% Основной текст
\section{Примеры нахождения суммы ряда}

\subsection{Задача 1}

\begin{enumerate}
    \item $\sum_{k=1}^{\infty} \frac{1}{(2k-1)(2k+1)}$
    \item $\sum_{k=1}^{\infty} \frac{1}{4k-2} - \frac{1}{4k+2} = \lim_{n \to \infty}0.5 - \frac{1}{4n+2} = 0.5$ 
\end{enumerate}


\subsection{Задача 2}


\subsection{Задача 3}
\begin{enumerate}
	\item $\sum_{1}^{\infty} \frac{2k+1}{k^2(k+1)^2}$
	\item $\sum_{1}^{\infty} \frac{1}{k^2}-\frac{1}{(k+1)^2} = \frac{1}{1} - \frac{1}{4} + \frac{1}{4} - \frac{1}{9} \dots = \lim_{k \to \infty} 1- \frac{1}{k} = 1$
\end{enumerate}

\subsection{Задача 4}
\begin{enumerate}
	\item $\sum_{k=2}^{\infty} \ln(1 - \frac{2}{k(k+1)}) = \ln(k-1) - \ln(k) - \ln(k+1)  + \ln(k+2) = -\ln(3)$ 
\end{enumerate}

\subsection{Задача 5}
\begin{enumerate}
	\item $\sum_{1}^{\infty} \sqrt{k+2}-2\sqrt{k+1}+\sqrt{k} = \sqrt{1}-2\sqrt{2}+\sqrt{3}+\sqrt{2}-2\sqrt{3} + \sqrt{4} + \sqrt{3} + 2\sqrt{4} + \sqrt{5} + \dots = 1 - \sqrt{2} + \frac{1}{\sqrt{1+n}-\sqrt{2+n}} = 1-\sqrt{2}$
\end{enumerate}

\subsection{Задача 6}
\begin{enumerate}
	\item $\sum_{k=1}^{\infty} \ln(k^2+3+\frac{2}{k^2})$
	\item $\lim_{k \to \infty} \ln(k^2+3+\frac{2}{k^2}) = \infty$ - таким образом не соответствует необходимому условию сходимости (предел к бесконечности не равен нулю, следовательно сумма бесконечно суммируется)
\end{enumerate}

\section{Критерий Коши сходимости числового ряда}

\subsection{Задача 7}
\begin{enumerate}
	\item $\forall \varepsilon > 0 \exists N \in \mathbb{N}: \forall n \ge N \forall p \in \mathbb{N} |\sum_{k=n+1}^{n+p} \frac{1}{\sqrt{k(k+1)}}|$
	\item Пусть $n  = p = N \Rightarrow |\sum_{k=N+1}^{2N}\frac{1}{\sqrt{k(k+1)}}| \ge |\frac{1}{\sqrt{2N(2N+1)}}| = \sum_{k=N+1}^{2N} \frac{1}{\sqrt{4+\frac{2}{N}}} < 0.5$ При N = 1 и $\varepsilon = \frac{1}{3}$ неравенство не выполняется, следовательно расходится. 
\end{enumerate}

\subsection{Задача 8}
\begin{enumerate}
	\item $\forall \varepsilon > 0 \exists N \in \mathbb{N}: \forall n \ge N \forall p \in \mathbb{N} |\sum_{k=1}^{\infty} \frac{\cos(kx)}{2^k}| < \varepsilon $
	\item $|\sum_{n=1}^{\infty} \frac{\cos(kx)}{2^k}| \le |\sum_{n=1}^{\infty} \frac{1}{2^k}|$
	\item Таким образом, т.к геометрический ряд, показатель которого меньше 1 сходится, то наша искомая функция сходится аболютно, а значит, ряд сходится.
\end{enumerate}

\subsection{Задача 9}
\begin{enumerate}
	\item $|\sum_{n=1}^{\infty} \frac{\ln(n)+\sin(n)}{n^2}| < \sum_{n=1}^{\infty} \frac{\ln(n)+1}{n^2}$
	\item Сделаем интегральный тест: $\int_{1}^{\infty} \frac{\ln(x)+1}{x^2}dx = (-\frac{\ln(n)+2}{n})|_{1}^{\infty} = 2$ - сходится, значит и наш ряд сходится.
\end{enumerate}

\subsection{Задача 10}
\begin{enumerate}
	\item $|\sum_{k=1}^{\infty} \frac{11+5(-1)^k}{2^{k+4}}|<\sum_{k=1}^{\infty} \frac{16}{2^{k+4}}$
	\item Воспользуемся интегральным тестом: $16 \int_{1}^{\infty} \frac{1}{2^{k+4}} = \frac{1}{\ln(4)}$
	\item Сходится абсолютно, значит и наш искомый интеграл сходится
	\item $\sum_{k=1}^{\infty} \frac{1}{k} + \frac{1}{2k^2}$ - не сходится т.к одно из слагаемых будет расходится
\end{enumerate}

\subsection{Задача 11}
\begin{enumerate}
	\item $\sum_{k=1}^{\infty} \frac{5k^5+e^k}{4^k+\ln(k+1)^3} < \sum_{k=1}^{\infty} \frac{5k^5+e^k}{4^k} \Rightarrow \sum_{k=1}^{\infty} \frac{5k^5}{4^k} + \frac{e^k}{4^k}$ - сходится т.к две суммы сходятся
	\item $\sum_{k=1}^{\infty} \frac{2k^3+7k+3}{\sqrt{k^8+6k^2+1}} < \sum_{k=1}^{\infty} \frac{2k^3}{k^4}$ Расходится.
\end{enumerate}

\subsection{Задача 12}
\begin{enumerate}
	\item $\sum_{k=1}^{\infty} (1-\ln(1+k)^{-\frac{1}{k}})$
	\item $\sum_{k=1}^{\infty} (1-e^{-\frac{1}{k}\ln(\ln(1+k))})$
	\item $\sum_{k=1}^{\infty} \frac{\ln(\ln(k))}{k}$ - расходится при интегральной оценке.
\end{enumerate}

\subsection{Задача 13}
\begin{enumerate}
	\item $\sum_{k=1}^{\infty} 1-\cos(\frac{\pi}{k^{1.5}})$
	\item Хочется попробовать через признак Д`Аламбера
	\item $\lim_{k \to \infty} \frac{a_{k+1}}{a_k} = \lim_{k \to \infty} \frac{1-\cos(\frac{\pi}{(k+1)^{1.5}})}{1-\cos(\frac{\pi}{k^{1.5}})}$
	\item Тейлором его около 0
	\item $\lim_{k \to \infty} \frac{a_{k+1}}{a_k} = \lim_{k \to \infty} \frac{\frac{\pi^2}{2(k+1)^3} - \frac{\pi^4}{24(k+1)^6}}{1-\frac{\pi^2}{2k^3}+\frac{\pi^4}{24k^6}} = \lim_{k \to \infty} \frac{k^3}{(k+1)} = 1$ - очень грустно
	\item $\sum_{k=1}^{\infty} \frac{\pi^2}{2k^3} < \pi^2 \sum_{k=1}^{\infty} \frac{1}{k^3}$ - сходится как известный p-ряд. Получается, что Даламбэро критерьеро не помог
\end{enumerate}

\end{document} 
