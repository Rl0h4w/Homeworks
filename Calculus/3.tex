\documentclass[a4paper,12pt]{article}

% Кодировка и язык
\usepackage[utf8]{inputenc}
\usepackage[russian]{babel}

% Математические пакеты
\usepackage{amsmath,amsfonts,amssymb}

% Графика
\usepackage{graphicx}
\usepackage{tikz}
\usetikzlibrary{shapes.geometric, calc}
\usepackage{pgfplots}
\pgfplotsset{compat=1.18} % Добавлено для устранения предупреждения

% Геометрия страницы
\usepackage{geometry}
\geometry{top=2cm, bottom=2cm, left=2.5cm, right=2.5cm}

% Гиперссылки
\usepackage{hyperref}

% Плавающие объекты
\usepackage{float}

% Дополнительные пакеты
\usepackage{venndiagram}

% Настройки заголовка
\title{Домашнее задание}
\author{Студент: \textbf{Ростислав Лохов}}
\date{\today}

\begin{document}

% Титульный лист
\begin{titlepage}
    \centering
    \vspace*{1cm}

    \Huge
    \textbf{Домашнее задание}

    \vspace{0.5cm}
    \LARGE
    По курсу: \textbf{Математический Анализ}

    \vspace{1.5cm}

    \textbf{Студент: Ростислав Лохов}

    \vfill

    \Large
    АНО ВО Центральный университет\\
    \vspace{0.3cm}
    \today

\end{titlepage}

% Содержание
\tableofcontents
\newpage

% Основной текст
\section{Производная и дифференциал вектор-функции}

\subsection{Задача 1}
\[
f(t)= g  \Rightarrow t = \frac{2\pi}{3}
\]

\[
f'(t) = \begin{pmatrix}
    -\sin(t) \\
    \cos(t) \\
    1
\end{pmatrix} = 
\begin{pmatrix}
    -\frac{\sqrt{3}}{2} \\
    -0.5 \\
    1
\end{pmatrix}
\]


\subsection{Задача 2}

\[
J = 
\begin{pmatrix}
    yz & xz & xy \\
    y(1-z) & x(1-z) & -xy \\
    -y & 1-x & 0 
\end{pmatrix}
\]

\[
\det(J) = xy^2
\]

\subsection{Задача 3}
\[
\begin{cases}
    u_1 = x_1 + x_2 + x_3 + \cdots  \\
    u_2 = 0.5x_1^2 + 0.5 x_2^2 + x_3^2 + \cdots \\
    u_3 = \frac{1}{3}x_1^3 + \frac{1}{3}x_2^3 + \frac{1}{3}x_3^3 + \cdots \\
    \cdots
\end{cases}
\]

\[
\frac{\partial u_i}{\partial x_j} = \frac{1}{i}\cdot i \cdot {x_j}^{i-1}={x_j}^{i-1}
\]

\[
\begin{pmatrix}
    {x_1}^{0} & {x_2}^{0} & {x_3}^{0} & \cdots \\
    {x_1}^{1} & {x_2}^{1} & {x_3}^{1} & \cdots \\ 
    {x_1}^{2} & {x_2}^{2} & {x_3}^{2} & \cdots \\
    \cdots 
\end{pmatrix}
\]

Это транспонированная матрица Вандермонда, ее определитель нам известен, но будет со знаком - из за транспонирования.

\[
\prod_{1\le i \le j \le n}^{n} (x_j - x_i)
\]

\subsection{Задача 4}
\[
J =
\begin{pmatrix}
    yz & xz & xy \\
    y-yz & x-xz & -xy \\
    -y & 1-x & 0 \\
\end{pmatrix} \cdot \begin{pmatrix}
    dx \\
    dy \\
    dz \\
\end{pmatrix} 
\]

\[
Det(J) = xy^2z(1-x)+x^2y^2z+xy^2(1-z) = xy^2
\]

\subsection{Задача 5}
\[
\nabla f(x, y) = 
\begin{pmatrix}
    \frac{2x}{y^2} \\
    \frac{-2x^2}{y^3}
\end{pmatrix}
\]

H(f) = 
\[
\begin{pmatrix}
    \frac{2}{y^2} & \frac{-4x}{y^3} \\
    \frac{-4x}{y^3} & \frac{6x^2}{y^4}
\end{pmatrix}
\]

\[
    H(f(1, 1)) = \begin{pmatrix}
    2 & -4 \\
    -4 & 6
    \end{pmatrix}
\]

\subsection{Задача 6}
\[
f(x) = \ln(\langle Ax; x\rangle)
\]

\[
f(x+dx) - f(x) = \ln(\langle Ax+dx; x+dx\rangle) - \ln(\langle Ax; x \rangle)
\]

\[
f(x+dx) - f(x) = \ln(\frac{\langle Ax+Adx; x+dx\rangle}{\langle Ax; x \rangle})
\]

\[
f(x+dx) - f(x) =  \ln(\frac{\langle Ax; x \rangle + \langle Ax; dx \rangle + \langle Adx; x \rangle + \langle Adx; dx \rangle }{\langle Ax; x \rangle})
\]

\[
f(x+dx) - f(x) =  \ln(1 + \frac{ \langle Ax; dx \rangle + \langle Adx; x \rangle + \langle Adx; dx \rangle}{\langle Ax; x \rangle})
\]

По эквивалентностям + опустим скалярное произведение $\langle Adx; dx \rangle$ т.к слишком мало

\[
f(x+dx) - f(x) =  \frac{ \langle Ax; dx \rangle + \langle Adx; x \rangle}{\langle Ax; x \rangle}
\]

Т.к оператор А симметричен, то:

\[
f(x+dx) - f(x) = 2 \frac{\langle Ax, dx \rangle}{\langle Ax; x \rangle}
\]

т.к $df = \langle \nabla f; dx \rangle$, то

\[
\nabla f = \frac{1}{\langle Ax; x \rangle} \nabla \langle Ax; x \rangle
\]

\[
d\langle Ax; x \rangle = \langle Ax + Adx; x + dx \rangle - \langle Ax; x \rangle
\]


\[
d\langle Ax; x \rangle = \langle Ax; x \rangle + \langle Ax; dx \rangle + \langle Adx; x \rangle + \langle Adx; dx \rangle - \langle Ax; x \rangle
\]

\[
d\langle Ax; x \rangle = \langle Ax; dx \rangle + \langle Adx; x \rangle = 2 \langle Ax; dx \rangle = \langle 2Ax; dx \rangle
\]

Отсюда следует, что $\nabla f = \frac{2Ax}{\langle Ax; x \rangle}$

Возьмем вспомогательную функцию $g(x) = df(x)$, тогда

\[
dg(x) = \langle \nabla df; dx \rangle = \langle \nabla (f(x+dx_1)-f(x)); dx \rangle
\]

\[
dg(x) =  \langle \nabla (f(x+dx)-f(x)); dx \rangle
\]

т.к функция линейна по аргументам, то 

\[
ddf(x) =  \langle \nabla f(x+dx)-\nabla f(x); dx \rangle
\]

\[
ddf(x) = \langle \frac{2A(x+dx)}{\langle A(x+dx); x+dx \rangle}-\frac{2Ax}{\langle Ax; x \rangle}; dx \rangle
\]

Лень сокращать, поэтому 

\[
H = \frac{2A(x+dx)}{\langle A(x+dx); x+dx \rangle}-\frac{2Ax}{\langle Ax; x \rangle}
\]

\subsection{Задача 7}
\begin{enumerate}
    \item бесконечное кол-во т.к $(x-y)(x+y) = 0$ для каждого x это уравнение выполняется, если $y=x$ или $y=-x$
    \item 4 - $y=x \land y=-x \land y=|x| \land y= -|x|$
    \item две - $y=x \land y=-x$ обе остальные недиф в нуле
    \item две  - $y=x \land y=|x|$
\end{enumerate}

\subsection{Задача 8}
\[
F_x' =2yu \land F_y' = 2xu \land F_u' = 3u^2+2xy 
\]

\[
\frac{\partial u}{\partial x} = -\frac{\partial F}{\partial x} \frac{\partial u}{\partial F} = 
\]

\[
\frac{\partial u}{\partial y} = -\frac{\partial F}{\partial y} \frac{\partial u}{\partial F}
\]

\[
u(0, 1) = -1
\]

\[
\frac{\partial u}{\partial x} = \frac{2}{3} \land \frac{\partial u}{\partial y} = 0
\]

\subsection{Задача 11}
\[
\begin{cases}
    z = u^3 + v^3 \\
    y = u^2 + v^2 \\
    x = u+v
\end{cases}
\]

\[
dz = 3u^2du + 3v^2dv \\
dy = 2udu + 2vdv \\
dx = du + dv 
\]

\[
du = dx - dv \\
\]

\[
dy = 2u(dx-dv) + 2vdv \\
\]

\[
dv = \frac{dy-2udx}{2(v-u)}
\]

\[
du = \frac{2vdx - dy}{2(v-u)}
\]

\[
dz = 3u^2\frac{2vdx - dy}{2(v-u)} + 3v^2\frac{dy-2udx}{2(v-u)}
\]

\[
dz = 1.5(-2uvdx+(u+v)dy)
\]

\[
uv=\frac{x^2-y}{2}
\]

\[
dz = 1.5(y-x^2)dx + 1.5xdy 
\]

\subsection{Задача 13}

\[
\frac{dy^2}{d^2x} + (x+y)(1+\frac{dy}{dx})^3 = 0
\]

\[
\frac{dx}{dt} = \frac{du}{dt} +1
\]

\[
\frac{dy}{dt} = \frac{du}{dt} -1
\]

\[
\frac{dx}{dy} = \frac{\frac{du}{dt}+1}{\frac{du}{dt}-1}
\]

\[
\frac{d^2x}{dy^2} = \frac{2u''}{(u'+1)^2}
\]

\[
\frac{2u''_{tt}(u'_{t}+1)+16u(u'_{t})^3}{(u'_{t}+1)^3}
\]

\subsection{Задача 14}

\[
\begin{cases}
    x = p\cos(\varphi) \\
    y = p \sin(\varphi)
\end{cases}
\]

\[  
\begin{cases}
    \frac{dx}{dt} = \cos(\varphi) \cdot \frac{dp}{dt} - \sin(\varphi)p\frac{d\varphi}{dt} \\
    \frac{dy}{dt} = \sin(\varphi)\frac{dp}{dt} + \cos(\varphi) p \cdot \frac{d\varphi}{dt}
\end{cases}
\]

\[
\begin{cases}
    \cos(\varphi)\frac{dp}{dt} - \sin(\varphi)p\frac{d\varphi}{dt} = p\sin(\varphi)+kp^3\cos(\varphi)\\
    \sin(\varphi)\frac{dp}{dt} + \cos(\varphi)p\frac{d\varphi}{dt} = -p\cos(\varphi)+kp^3\sin(\varphi)
\end{cases}
\]

\[
\begin{cases}
    \frac{d\varphi}{dt} = -1
    \frac{dp}{dt} = kp^3
\end{cases}
\]

\subsection{Задача 15}

\[
(y-z)\frac{dz}{dx} + (y+z)\frac{dz}{dy} = 0 
\]

\[
\begin{cases}
    u = y-z \\
    v = y + z 
\end{cases}
\]

\[
\frac{du}{dx} = 0 \land \frac{dv}{dx} = 0 \land \frac{du}{dy} = 1 \land \frac{dv}{dy} = 1
\]

\[
\frac{dz}{dy} = \frac{dz}{du}\frac{du}{dy} + \frac{dz}{dv}\frac{dv}{dy}
\]

\[
\begin{cases}
    \frac{dz}{du} = -0.5 \\
    \frac{dz}{dv} = 0.5
    \frac{dz}{dy} = 0
\end{cases}
\]



\end{document}