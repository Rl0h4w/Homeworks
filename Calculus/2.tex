\documentclass[a4paper,12pt]{article}

% Кодировка и язык
\usepackage[utf8]{inputenc}
\usepackage[russian]{babel}

% Математические пакеты
\usepackage{amsmath,amsfonts,amssymb}

% Графика
\usepackage{graphicx}
\usepackage{tikz}
\usetikzlibrary{shapes.geometric, calc}
\usepackage{pgfplots}
\pgfplotsset{compat=1.18} % Добавлено для устранения предупреждения

% Геометрия страницы
\usepackage{geometry}
\geometry{top=2cm, bottom=2cm, left=2.5cm, right=2.5cm}

% Гиперссылки
\usepackage{hyperref}

% Плавающие объекты
\usepackage{float}

% Дополнительные пакеты
\usepackage{venndiagram}

% Настройки заголовка
\title{Домашнее задание}
\author{Студент: \textbf{Ростислав Лохов}}
\date{\today}

\begin{document}

% Титульный лист
\begin{titlepage}
    \centering
    \vspace*{1cm}

    \Huge
    \textbf{Домашнее задание}

    \vspace{0.5cm}
    \LARGE
    По курсу: \textbf{Математический Анализ}

    \vspace{1.5cm}

    \textbf{Студент: Ростислав Лохов}

    \vfill

    \Large
    АНО ВО Центральный университет\\
    \vspace{0.3cm}
    \today

\end{titlepage}

% Содержание
\tableofcontents
\newpage

% Основной текст
\section{Частные производные}

\subsection{Задача 1}

a)
\[
\frac{\partial f}{\partial x} = \frac{3\arctan(x)^2}{(1+x^2)}
\]

\[
\frac{\partial f}{\partial y} = \frac{6\arcsin(2y)^2}{\sqrt{1-4y}}
\]

b)
\[
\frac{\partial f}{\partial x} = \frac{2x-y}{y^2}
\]

\[
\frac{\partial f}{\partial y} = \frac{-2x^2+xy}{y^3}
\]

c) 
\[
\frac{\partial f}{\partial y} = 6x \cos(6x \tan(y)) \sec(y)^2 - \frac{x^2}{6\ln(5)y}
\]

\[
\frac{\partial f}{\partial x} = 6\cos(6x\tan(y))\tan(y) - 2\log_{5}(\sqrt[6]{11y})x
\]

d)
\[
\frac{\partial f}{\partial x} = \frac{-2x\cos(3y)+2x^2\sin(y)^4-sin(y)^4}{e^{x^2}}
\]

\[
\frac{\partial f}{\partial y} = 6x\cos(6x\tan(y))\sec(y)^2-\frac{x^2}{6\ln(5)y}
\]

\subsection{Задача 2}

a)
\[
\frac{\partial f}{\partial x} = \frac{e^{\arctan(xy^3)}y^3}{1+x^2y^6}= \frac{1}{2} e^{\pi/4}
\]

\[
\frac{\partial f}{\partial y} = \frac{3e^{\arctan(xy^3)}xy^2}{1+x^2y^6} = \frac{1}{2} e^{\pi/4}
\]

b)
\[
\frac{\partial f}{\partial x} = (3x+y)^{3x+y}(3+3\ln(3x+y)) = 3
\]

\[
\frac{\partial f}{\partial y} = (3x+y)^{3x+y}(1+\ln(3x+y)) = 1
\]

\subsection{Задача 3}

\[
\sqrt{|xy|} \leftrightarrow 0 \le \sqrt{|x|}\sqrt{|y|} \le \frac{|x|+|y|}{2}
\]
по теореме о пределе промежуточной функции, сжатия, теореме о двух милиционерах предел нашей функции будет равен значению в 0, тогда функция непрерывна.


\[
f_x(0, 0) = \lim_{h \to 0} \frac{f(h, 0)-f(0, 0)}{h} = \lim_{h \to 0}\frac{0}{h} = 0
\]

\[
f_y(0, 0) = \lim_{h \to 0} \frac{f(0, h)-f(0, 0)}{h} = \lim_{h \to 0}\frac{0}{h} = 0
\]

Таким образом в точке (0, 0) существуют обе частные производные.

функция назвается дифференцируемой в точке, если 
\[
\Delta f(x, y) = \frac{\partial f}{\partial x}(x_0, y_0)\Delta x + \frac{\partial f}{\partial }(x_0, y_0)\Delta y + o(p)
\]

Подставляем, получаем
\[
f(x, y) = o(\sqrt{x^2+y^2})
\]

подставим y=x
\[
|x| = \sqrt{2}|x|
\]
Значит не дифференцируема.


\subsection{Задача 5}

Пользуясь эквивалентностями
\[
f(x, y) = |x|^a|y|^{0.5}
\]

Перейдем к полярным координатам: 

\[
f(a, \theta) = r^{\alpha+0.5}|\cos(\theta)|^\alpha|\sin(\theta)|^{0.5}
\]

При r стремящемся к 0 функция стремится к 0, если 
\[
\alpha > -0.5
\]

При отрицательных $\alpha$ возникает деление на ноль, поэтому  функция имеет смысл только при $\alpha > 0$

Тогда функция непрерывна при $\alpha \ge 0$ по определению

Найдем частные производные
\[
f_x(0, 0) = \lim_{h \to 0} \frac{f(h, 0)-f(0, 0)}{h} = 0
\]

\[
f_y(0, 0) = \lim_{h \to 0} \frac{f(0, h)-f(0, 0)}{h} = 0
\]

таким образом, если функция дифференцируема, то ее дифференциал должен быть равен в точках 0, 0.

\[
\lim_{(x, y) \to (0, 0)} \frac{f(x, y)-0}{\sqrt{x^2+y^2}} = r^{\alpha-0.5}|\cos(\theta)|^\alpha|\sin(\theta)|^{0.5}
\]

если a > 0.5, то дифференцируема.

\subsection{Задача 6}

\[
\Delta f(x, y) = \frac{\partial f}{\partial x}(x_0, y_0)\Delta x + \frac{\partial f}{\partial }(x_0, y_0)\Delta y + o(p)
\]

Найдем частные производные

\[
\frac{d}{dx}\left(\frac{\sqrt{|xy|} \arctan\left(\sqrt{x^2 + y^2}\right)}{\left(x^4 - \frac{x^2 y^2}{3} + y^4\right)^{1/6}}\right) =
\]
\[
\frac{1}{2 \left(3 x^4 - x^2 y^2 + 3 y^4\right)^{7/6} |xy|^{3/2}} \times
\]
\[
\left( 3^{1/6} \left(x y^2 \left(3 x^4 - x^2 y^2 + 3 y^4\right) \arctan\left(\sqrt{x^2 + y^2}\right) - \frac{2}{3} x^3 y^2 \left(6 x^2 - y^2\right) \arctan\left(\sqrt{x^2 + y^2}\right) \right. \right.
\]
\[
\left. \left. + \frac{2 x^3 y^2 \left(3 x^4 - x^2 y^2 + 3 y^4\right)}{\sqrt{x^2 + y^2} \left(x^2 + y^2 + 1\right)} \right) \right)
\]

\[
\frac{\partial}{\partial y}\left(\frac{\sqrt{|xy|} \arctan\left(\sqrt{x^2 + y^2}\right)}{\left(x^4 - \frac{x^2 y^2}{3} + y^4\right)^{1/6}}\right) =
\]
\[
\frac{y \sqrt{|xy|}}{\sqrt{x^2 + y^2} \left(x^2 + y^2 + 1\right) \left(x^4 - \frac{x^2 y^2}{3} + y^4\right)^{1/6}} +
\]
\[
\frac{x^2 y \arctan\left(\sqrt{x^2 + y^2}\right)}{2 \left(x^4 - \frac{x^2 y^2}{3} + y^4\right)^{1/6} |xy|^{3/2}} -
\]
\[
\frac{\left(4 y^3 - \frac{2 x^2 y}{3}\right) \sqrt{|xy|} \arctan\left(\sqrt{x^2 + y^2}\right)}{6 \left(x^4 - \frac{x^2 y^2}{3} + y^4\right)^{7/6}}
\]

предел к 0 обоих производных даёт 0. тогда:

\[
\lim_{(x, y) \to (0, 0)} \frac{f(x, y)-0}{\sqrt{x^2+y^2}}
\]

Перейдем в полярные координаты:

\[
\frac{r^{1/3} \arctan(r) \sqrt{|\sin(2\theta)|}}{\sqrt{2} \left(1 - \frac{7}{12} \sin^2(2\theta)\right)^{1/6}} = 0
\]

таким образом можно сделать оценку для нашей функции, что она стремится к 0 не больше, чем $r^\frac{2}{3}$, и отсюда следует, что она дифференцируема

\subsection{Задача 8}

\[
\frac{\partial f}{\partial x} = e^{x^2 y+\pi \cos x} \cdot (2xy - \pi \sin x)
\]

\[
\frac{\partial f}{\partial y} = e^{x^2 y+\pi \cos x} \cdot (x^2)
\]

\[
df = e^{-\pi/2+\pi \cos 1} \left[ -\pi(1 + \sin 1) dx + dy \right]
\]

\subsection{Задача 9}
\[
z = u^2/v
\]

\[
d(\arctan(z)) = \frac{1}{1+z^2}dz
\]

\[
d(f/g) = (g df - f dg)/g^2
\]

\[
dz = d(u^2/v) = (vd(u^2)-u^2d(v))/v^2
\]

\[
dz = (2uv du - u^2 dv) / v^2
\]

\[
d(\arctg(u^2/v)) = (1 / (1 + (u^2/v)^2)) * [(2uv du - u^2 dv) / v^2]
\]

\[
d(\arctg(u^2/v)) = (2uv du - u^2 dv) / (v^2 + u^4)
\]

\subsection{Задача 11}
\[
\frac{\partial u}{\partial x} = \frac{\partial}{\partial x} (xy) = y
\]
\[
\frac{\partial u}{\partial y} = \frac{\partial}{\partial x} (xy) = x
\]
\[
\frac{\partial v}{\partial x} = \frac{\partial}{ \partial x} (x^2-y^2) = 2x
\]
\[
\frac{\partial v}{\partial y} = -2y
\]

\[
\frac{\partial f}{\partial x} = y\frac{\partial f}{\partial u} + 2x \frac{\partial f}{\partial v}
\]

\[
\frac{\partial f}{\partial x} = x \frac{\partial f}{ \partial u} - 2y \frac{\partial f}{\partial v}
\]

\subsection{Задача 12} 



\begin{align*}
\frac{\partial u}{\partial x} &= \frac{\partial}{\partial x} \left(\frac{x}{y}\right) = \frac{1}{y} \\
\frac{\partial u}{\partial y} &= \frac{\partial}{\partial y} \left(\frac{x}{y}\right) = -\frac{x}{y^2} \\
\frac{\partial u}{\partial z} &= \frac{\partial}{\partial z} \left(\frac{x}{y}\right) = 0
\end{align*}

\begin{align*}
\frac{\partial v}{\partial x} &= \frac{\partial}{\partial x} \left(x^2 + y - z^2\right) = 2x \\
\frac{\partial v}{\partial y} &= \frac{\partial}{\partial y} \left(x^2 + y - z^2\right) = 1 \\
\frac{\partial v}{\partial z} &= \frac{\partial}{\partial z} \left(x^2 + y - z^2\right) = -2z
\end{align*}


\begin{align*}
\varphi'_x = \frac{\partial \varphi}{\partial x} &= \frac{\partial f}{\partial u} \frac{\partial u}{\partial x} + \frac{\partial f}{\partial v} \frac{\partial v}{\partial x} = f'_u \cdot \frac{1}{y} + f'_v \cdot 2x \\
\varphi'_y = \frac{\partial \varphi}{\partial y} &= \frac{\partial f}{\partial u} \frac{\partial u}{\partial y} + \frac{\partial f}{\partial v} \frac{\partial v}{\partial y} = f'_u \cdot \left(-\frac{x}{y^2}\right) + f'_v \cdot 1 \\
\varphi'_z = \frac{\partial \varphi}{\partial z} &= \frac{\partial f}{\partial u} \frac{\partial u}{\partial z} + \frac{\partial f}{\partial v} \frac{\partial v}{\partial z} = f'_u \cdot 0 + f'_v \cdot (-2z) = -2z f'_v
\end{align*}


\begin{align*}
&2xz\varphi'_x + 2yz\varphi'_y + (2x^2 + y)\varphi'_z \\
&= 2xz \left(f'_u \cdot \frac{1}{y} + f'_v \cdot 2x\right) + 2yz \left(f'_u \cdot \left(-\frac{x}{y^2}\right) + f'_v \cdot 1\right) + (2x^2 + y) \left(-2z f'_v\right) \\
&= \frac{2xz}{y} f'_u + 4x^2z f'_v - \frac{2xyz}{y^2} f'_u + 2yz f'_v - (4x^2z + 2yz) f'_v \\
&= \frac{2xz}{y} f'_u + 4x^2z f'_v - \frac{2xz}{y} f'_u + 2yz f'_v - 4x^2z f'_v - 2yz f'_v \\
&= \left(\frac{2xz}{y} - \frac{2xz}{y}\right) f'_u + \left(4x^2z - 4x^2z\right) f'_v + \left(2yz - 2yz\right) f'_v \\
&= 0 \cdot f'_u + 0 \cdot f'_v + 0 \cdot f'_v \\
&= 0
\end{align*}

\subsection{Задача 14}
\[
\frac{\partial f}{\partial x} = -\frac{x}{(x^2+y^2+z^2)^{\frac{3}{2}}} = -\frac{1}{14\sqrt(14)}
\]

\[
\frac{\partial f}{\partial x} = -\frac{x}{(x^2+y^2+z^2)^{\frac{3}{2}}} = -\frac{2}{14\sqrt(14)}
\]

\[
\frac{\partial f}{\partial x} = -\frac{x}{(x^2+y^2+z^2)^{\frac{3}{2}}} = -\frac{3}{14\sqrt(14)}
\]

\[
\nabla f(1;2;3) = -14\sqrt{14} (1;2;3)
\]

\end{document}